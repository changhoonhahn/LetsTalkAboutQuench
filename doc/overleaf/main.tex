\documentclass[preprint2,tighten]{aastex62}
\pdfoutput=1 %for arXiv submission
\usepackage{amsmath,amstext}
\usepackage[T1]{fontenc}
\usepackage{txfonts} %use times font for math
\usepackage[figure,figure*]{hypcap} %Figure refs go figures
\usepackage{bm}
\usepackage{chngcntr}

\renewcommand*{\sectionautorefname}{Section} %for \autoref
\renewcommand*{\subsectionautorefname}{Section} %for \autoref

\newcommand*{\kms}{\,km~s$^{-1}$}
\newcommand{\todo}[1]{{\bf \textcolor{red}{ #1}}}

\shorttitle{}
\shortauthors{the IQ (Isolated \& Quiescent) collaboratory}

\begin{document}

\title{IQ-Collaboratory 1.1: the Star Forming Sequence of Simulated Galaxies}
%\title{Distributions of star-forming galaxies: predictions}
\author{ChangHoon Hahn}
\altaffiliation{changhoonhahn@lbl.gov}
\affil{Lawrence Berkeley National Laboratory, 1 Cyclotron Rd, Berkeley CA 94720, USA}
\affil{Berkeley Center for Cosmological Physics, University of California, Berkeley, CA 94720, USA}
\author{Tjitske K. Starkenburg}
\affil{Flatiron Institute, 162 Fifth Avenue, New York NY 10010, USA}
\author{Claire Dickey}
\affil{Yale University}

\author{IQ collaboratory (in alphabetical order)}
\affil{various}

\begin{abstract}
Principled comparisons between the galaxy populations of simulations and 
observations play a pivotal role in validating our theories of galaxy 
formation and evolution. %are necessary to validate our theories of galaxy formation and  evolution. 
Key features in galaxy property-space like the star forming sequence
(SFS) %,\todo{tjitske I think this subsentence should be removed: \emph{which track %encapsulate %the evolution of  star forming galaxies since $z<2$,} because it is not as essential and it kind of steals away attention from the rest of the sentence which is actually the main point you want to make. Also, you say the z<2 part clearly in the introduction}
and a consistent way to identify them are critical for such comparisons. 
We present a data-driven approach to fitting the SFS using Gaussian 
Mixture Model that can be flexibly applied to a wide range of star 
formation to stellar mass relations down to $M_*{\sim}10^{8}M_\odot$. %and across four orders of magnitude in stellar mass.
Using this method, we identify the SFS of central galaxies in 
the Illustris, EAGLE, and {\sc Mufasa} hydrodynamic simulations, the Santa Cruz 
Semi-Analytic Model, and observations from the Sloan Digital 
Sky Survey Data Release 7 and the NASA Sloan Atlas. Among our simulations, 
we find order of magnitude discrepancies in the amplitudes of the SFSs. 
In addition to the SFS, our fitting method also identifies subpopulations 
that correspond to quiescent, transitioning, and star-burst galaxies. 
We find overall consistent subpopulations among our hydrodynamic simulations 
%using subpopulations such as quiescent galaxies, also identified by our fitting method, we find  % also identifies subpopulations  that correspond to quiescent, transitioning, and star-burst galaxies. We 
but differences with the semi-analytic model. We also find that the quiescent 
fractions of the hydrodynamic simulations do not reproduce observations. 
Moreover, in \emph{all} of the simulations we find a significant fraction 
of quiescent galaxies at $M_*{<}\,10^9M_\odot$, in conflict with the literature. %below the established \cite{geha2012} threshold. 
The SFS fitting method we present provides a data-driven framework to 
consistently compare galaxy samples from both simulations and observations. 
\end{abstract}
\keywords{cosmology: observations --- galaxies: star formation --- galaxies:statistics}

\section{Introduction}
Large galaxy surveys of the past decade such as the Sloan Digital Sky 
Survey~\citep[SDSS;][]{york2000}, have firmly established the major 
trends of galaxies in the local universe. Galaxies %in the properties 
broadly fall into two populations: quiescent galaxies with little star
formation that are red in color with elliptical morphologies and star 
forming galaxies with significant star formation that are blue in color 
with disk-like morphologies 
(\citealt{kauffmann2003, blanton2003, baldry2006, taylor2009, moustakas2013}; 
for a recent review see~\citealt{blanton2009}). 
Star forming galaxies, furthermore, are found to have a tight relationship 
between their star formation rates (SFR) and stellar masses placing them
on the so-called ``star formation sequence'' (hereafter 
SFS)~\citep[\emph{e.g.}][]{noeske2007, daddi2007, salim2007}.

%Star forming galaxies are found to have tightly correlated 
%star formation rates (SFR) and stellar 
%masses~\citep[\emph{e.g.}][\todo{more}]{noeske2007, daddi2007, salim2007}.

% Galaxy evolution from the perspective of the SFS. 
In fact, this sequence of star forming galaxies is found in observations 
well beyond the local universe out to $z > 2$~\citep{wang2013, schreiber2015}.
But more than its persistence, the SFS plays a crucial role in characterizing 
the evolving galaxy population~\citep[but see also][]{leja2015}. The most dramatic transformations of 
galaxies over the past $10\,\mathrm{Gyr}$ can be described by the SFS. 
For instance, the decline in the number density of massive 
star forming galaxies and the accompanying growth in number density of 
quiescent galaxies reflects the cessation of star formation in 
star forming galaxies migrating off of the 
SFS~\citep{blanton2006, borch2006, bundy2006, moustakas2013}. 
Similarly, the cosmic decline in star formation~\citep{hopkins2006,
behroozi2013a, madau2014} reflects the overall decline of star 
formation of the SFS~\citep{schreiber2015}. 

%in the continuing effort to understand  the physical processes governing galaxy formation and evolution, 
Numerical simulations today \emph{qualitatively} reproduce the SFS and 
similar global relations of galaxy properties and provide insights into the key physical processes governing those relations 
(\emph{e.g.}~\citealt{ vogelsberger2014,genel2014, schaye2015, dave2017}; 
for a recent review see~\citealt{somerville2015}). These hydrodynamic and 
semi-analytic simulations each seek to capture the complex physics of 
gas heating and cooling, star formation, stellar feedback, chemical 
evolution, black hole formation and evolution, and AGN feedback using 
their distinct sub-grid model prescriptions. Both as an effort to shed 
light on the underlying physics and to validate their simulations and make predictions for new observations, many 
works have already compared the simulations considered in this paper to 
observations of for example galaxy masses, colors, and star formation rates ~\citep[\emph{e.g.}][]{vogelsberger2014, 
genel2014, torrey2014, sparre2015, schaye2015, bluck2016, dave2017, somerville2015}. 
These works, however, primarily focus on comparing one specific simulated 
galaxy sample to one or a few observational datasets (which can differ significantly as well as demonstrated in~\citealt{speagle2014}). Extending such 
comparisons to include multiple simulations, observations, and a 
consistent framework for comparing the data-sets would allow us to make 
detailed comparison of the different sub-grid models and thereby provide 
key insights into the physics that govern galaxy formation and evolution.  

%At the same time a general understanding of the key physical processes  in the formation and evolution of galaxies has arisen from numerical simulations. Global galaxy properties, correlations, and distribution  functions can nowadays (approximately) be reproduced by hydrodynamic,  and semi-analytical cosmological and zoom simulations that include gas  heating and cooling, star formation and stellar feedback, chemical  evolution, and black hole formation, evolution, and AGN feedback. Due to the immense range in scales that needs to be covered by these simulations, many  of these processes are implemented using sub-grid models, that vary in their prescriptions depending on simulation method, code, and resolution. Even  though the effects can be similar, the detailed processes in these feedback schemes may differ, for example whether feedback due to AGN in simulations  is most effective in suppressing cold gas accretion and star formation when modeled as thermal, kinetic, or radiative feedback, or any combination of  these depending on central black hole and galaxy properties \todo{refs}.  Present-day galaxy formation models do qualitatively reproduce the evolution of the SFS and similar global scaling relations \citep{SomervilleDave2015ARAA, vogelsberger2014,genel2014, schaye2015, dave2016, dave2017}\todo{and more}. Many comparisons of the SFS between observations and simulations have been made, but these usually focus on comparing one specific simulated galaxy sample to one or a few observational datasets \citep{vogelsberger2014,genel2014, schaye2015, dave2016,bluck2016, torrey2014,sparre2015}\todo{and more}. 

%Nonetheless, with the key role that starforming galaxies play in our understanding of the formation and evolution of galaxies, the SFS is an important relation to compare to for simulations. Moreover, the SFS can form a tool to understand the starforming and non-starforming galaxy populations and the processes that create those. Therefore we argue that a comparing the galaxy populations in the SFR-$M_{\star}$ plane between different simulations that vary in technique and in descriptions of key physical processes, is important. The SFS serves as a key feature in galaxy sample data for comparison. Therefore in this paper we compare different simulations by conducting a data-driven comparison centered around the star formation main  sequences of different data sets. 
The SFS, given its prominence, naturally presents itself as a key feature 
in the data-space of galaxy properties to compare galaxy populations across 
both observations and simulations. Moreover, with the important role it plays 
for understanding galaxy evolution, the SFS provides a way to understand the 
different galaxy subpopulations and the processes that create them. Two main 
challenges lie in conducting a principled comparison of the SFS. First is the 
lack of a consistent and data-driven method for identifying the SFS given a 
dataset of galaxy properties. {\color{red} In fact, previously found agreement 
among the SFSs of simulations and observations~\citep[\emph{e.g.}][]{somerville2015b} 
is primarily driven by discrepancies in how the SFS is identified (Appendix~\ref{app:literature}).}
The other is the difference in methodology for 
deriving galaxy properties (such as SFR, $M_*$) in different data-sets, which 
alone dramatically impacts the SFS~\citep[\emph{e.g.}][]{speagle2014}. 
In this paper we address the first challenge by presenting a flexible, 
data-driven method for fitting the SFS. Then we use this method to conduct
a principled comparison between the central galaxy populations from the Illustris,
EAGLE, and {\sc Mufasa} hydrodynamic simulations, the Santa Cruz Semi-Analytic 
Model (SAM) simulation and observations from the SDSS and NASA-Sloan Atlas
(hereafter NSA) catalogs. 

In Section~\ref{sec:galsims}, we describe the simulated and observed 
galaxies of our dataset and how we specifically select our galaxy sample. 
Then in Section~\ref{sec:sfmsfit}, we describe our data-driven SFS 
fitting method, which makes use of Gaussian Mixture Modeling. We present 
the results from applying our SFS fitting to our simulated and observed
galaxies in Section~\ref{sec:results} and compare the galaxy populations
of our simulations and observations. Finally, we conclude and summarize
the results of our comparison in Section~\ref{sec:summary}.
This paper is the first in a series studying the star formation and 
quenching properties of galaxies. The series is initialized by the 
IQ (Isolated \& Quiescent) Collaboratory. %This first paper will focus on the rate of star formation in observed and simulated galaxies. 
In the subsequent paper we will address the discrepancies in measured 
galaxy properties by constructing mock observational spectra of simulated 
galaxies in our sample and measuring the properties of these galaxies
in the same manner as observations (Starkenburg et al. in prep).
%for comparing the galaxy populations in simulations with observations.  One challenge is however not one unified way to fit and describe the SFS, and depending on both the dataset and the fitting technique used its properties can vary \citep[see e.g.][]{speagle2014}, adding to the complexity of the discussion on its meaning.

%%%%%%%%%%%%%%%%%%%%%%%%%%%%%%%%%%%%%%%%%%
% Figure 1 
%%%%%%%%%%%%%%%%%%%%%%%%%%%%%%%%%%%%%%%%%%
\begin{figure*}
\begin{center}
\includegraphics[width=\textwidth]{figs/Catalogs_SFR_Mstar.pdf} 
\caption{The SFR--$M_*$ relations of central galaxies from the Illustris, 
EAGLE, {\sc Mufasa}, and SC-SAM simulations (left to right). The 
top panels use instantaneous SFRs while the bottom panels use SFRs 
averaged over $100\,\mathrm{Myr}$. The simulations and how they derive 
the SFRs are described in Section~\ref{sec:galsims}. Although a
direct comparison to observations is tenuous due to the fact that 
the SFRs and $M_*$s of the observed SDSS galaxies are \emph{not} 
derived consistently as simulations, we include, for reference, the 
observed  SDSS galaxies (Section~\ref{sec:obvs}) on the right. 
\emph{The SFR--$M_*$ relations in every panel reveals a clear star 
forming sequence.}} 
\label{fig:sfrmstar}
\end{center}
\end{figure*}
%%%%%%%%%%%%%%%%%%%%%%%%%%%%%%%%%%%%%%%%%%

%%%%%%%%%%%%%%%%%%%%%%%%%%%%%%%%%%%%%%%%%%
% Figure  
%%%%%%%%%%%%%%%%%%%%%%%%%%%%%%%%%%%%%%%%%%
\begin{figure}
\begin{center}
\includegraphics[width=0.475\textwidth]{figs/Catalogs_pSSFR.pdf} 
\caption{The SSFR distributions, $p(\log\,\mathrm{SSFR})$, of the 
central galaxies in the Illustris (green), EAGLE (red), 
{\sc Mufasa} (purple), and  SC-SAM (brown) simulations with 
$10.4 < \log\,M_* < 10.6$. We use instantaneous SFRs on the left
and SFRs averaged over $100\,\mathrm{Myr}$ on the right. Simulated
galaxies with $\mathrm{SFR}{=}0$ are assigned $\log\,\mathrm{SSFR}{=}-13.2$ 
so that they can be included in the figure. Although the SFS is 
universal (Figure~\ref{fig:sfrmstar}), \emph{the significant 
discrepancies among the $p(\log\,\mathrm{SSFR})$s make the SFS 
difficult to consistently quantify.}} \label{fig:pssfr}
\end{center}
\end{figure}
%%%%%%%%%%%%%%%%%%%%%%%%%%%%%%%%%%%%%%%%%%

\section{Our Simulated Galaxies} \label{sec:galsims}% \label{sec:ourgals}
In this work, our main focus is to compare simulated central galaxies 
from four large-scale cosmological simulations: three hydrodynamic 
(Illustris, EAGLE, and {\sc Mufasa}) and one semi-analytic (SC-SAM). 
A consistent comparison of the different galaxy populations requires 
consistently defined galaxy properties. Therefore, for all of our 
simulated galaxies, we derive their stellar masses using the same 
definition and SFRs on two timescales: instantaneous and averaged over 
$100\,\mathrm{Myr}$. We choose these timescales because they correspond
to $H{\alpha}$ and $UV$ based SFR measurements, which represent the 
formation of young stars with ages ${\lesssim}10\,\mathrm{Myr}$ 
and star formation in the last ${\sim}100\,\mathrm{Myr}$~\citep[e.g.][]{kennicutt2012}, 
respectively. 

In our hydrodynamic simulations, we derive the $100\,\mathrm{Myr}$ 
averaged SFRs from the ages, or formation times, of star particles 
in the galaxies and instantaneous SFRs from the rate of 
star formation in the dense and cold gas. For the SAM, we derive 
the $100\,\mathrm{Myr}$ averaged SFRs from the total stellar 
mass formed in the galaxies, which is outputted from the model 
every $10\,\mathrm{Myr}$. The instantaneous SFR, we derive using 
the Kennicutt-Schmidt relation for molecular hydrogen~\citep{bigiel2008} 
and the derived H$_2$ surface density in radial bins. The spatial, 
mass, and temporal resolution of our simulation can significantly 
impact the SFRs --- particularly the $100\,\mathrm{Myr}$ averaged 
SFRs in the hydrodynamic simulations. In Appendix~\ref{app:zerosfr}, 
we discuss how we treat these resolution effects in our analysis. 
Meanwhile, for the stellar mass of our simulated galaxies, we 
measure the total stellar mass within the host halos, discounting the 
stellar mass in any subhalo within the halo. Although halos are 
identified differently in the simulations, nearly all the stellar 
mass is in the center of halos. Therefore, the stellar masses are
consistently defined among our simulations.

In the rest of this section we provide a brief description of the 
Illustris, EAGLE, {\sc Mufasa}, and Santa-Cruz SAM simulations and each 
of their key sub-grid and feedback prescriptions. Furthermore, in 
Section~\ref{sec:obvs}, we briefly describe galaxy samples from 
observation, a volume-limited sample from SDSS and a selected isolated 
dwarf galaxy sample from the NSA catalog, which we include for reference.
Lastly in Section~\ref{sec:central}, we describe how we consistently 
identify central galaxies among our simulations and observations. 

%We compare the population of the SFR--$M_{\star}$ plane between the different large-scale cosmological simulations and the observational datasets. To ensure that we compare similar star formation rate timescales, we use averaged star formation rates over chosen recent timescales for all the simulated galaxies. For our analysis, however, we use SFRs averaged 
%over $100\,\mathrm{Myr}$ and instantaneous SFR, to be roughly consistent with respectively observed SFRs based on $H{\alpha}$ observations, corresponding to young stars with ages $\lesssim 10\,\mathrm{Myr}$, and observed SFRs based on $UV$ brightness, corresponding to stars formed in the last $\sim 100\,\mathrm{Myr}$ \citep[e.g.][]{kennicutt2012}. 

%and how  we derive the instantaneous and $100\,\mathrm{Myr}$ averaged SFRs for  each simulation.

%For the galaxies in the simulations, we compare star formation rates averaged over the last $10\,\mathrm{Myr}$, $100\,\mathrm{Myr}$, and $1\,\mathrm{Gyr}$, as well as the instantaneous SFR. 
%We note that spatial and temporal resolution effects in the simulations  can cause averaged SFRs from stellar ages to under-predict the actual  averaged SFRs in simulations. 
%(most notably the $10\,\mathrm{Myr}$--averaged SFR) can strongly under-predict the actual averaged SFR in the simulations. We therefore add conservative upper limits and uncertainties to these values and study the effect of these on our results.

%Our figures will mostly focus on comparisons of the observationally most relevant timescales (instantaneous, $10\,\mathrm{Myr}$ and $100\,\mathrm{Myr}$) but we will comment on deviations for longer or shorter timescales.

\subsection{The Illustris simulation}
The Illustris simulation \citep{vogelsberger2014,genel2014} evolves a
cosmological volume of $(106\ \rm{Mpc})^3$ with a uniform baryonic mass
resolution of $1.6\times10^6M_{\sun}$ using the Arepo moving-mesh 
code~\citep{springel2010}. 
The full halo mass range of $2 \times 10^8$ to $3\times 10^{14}\ M_{\sun}$ 
is set by resolution (32 particles) at the low-mass end and by volume at 
the high-mass end (e.g.~including 10 halos with $M>10^{14}M_{\sun}$). 
It employs sub-grid models for star-formation~\citep{springel2003},
Bondi-like SMBH accretion, a phenomenological model for galactic winds, and 
two main modes for energy injection from SMBHs~\citep[\emph{see}][]{vogelsberger2013}. 
When accretion occurs at Eddington ratios $>0.05$, thermal energy is 
injected continuously in the local environment of the SMBH, while at 
lower accretion rates, the energy injection occurs in bursts at large
distances from the SMBH, generating hot bubbles in the ICM~\citep{sijacki2007}.
%The latter mode of feedback is responsible for a steep drop in the  cosmic star-formation rate density at late times~\citep{vogelsberger2013}.
Previous works discussing aspects of the star-formation main-sequence and/or
quenching in the Illustris simulation include 
\citet{vogelsberger2014, sparre2015, bluck2016, terrazas2017}.

\subsection{EAGLE}
The Virgo Consortium's Evolution and Assembly of GaLaxies and their 
Environment (EAGLE) project~\citep{schaye2015, crain2015} is a suite 
of cosmological, hydrodynamic simulations of a standard 
$\Lambda$ cold dark matter universe using $\mathtt{ANARCHY}$ (Dalla 
Vecchia et al. in prep.; see also Appendix A of \citealt{schaye2015} and 
\citealt{schaller2015}). $\mathtt{ANARCHY}$ is a modified version of 
the $\mathtt{Gadget}$ 3 $N$-body/SPH code ~\citep{springel2005} that
includes modifications to the SPH formulation, time stepping, and sub-grid 
physics. The sub-grid model for feedback from massive stars and AGN is 
based on thermal energy injection in the ISM without the need to turn-off
cooling or hydrodynamic decoupling of winds~\citep{dallavecchia2012}. 
Similar to semi-analytical models, the sub-grid parameters for stellar 
feedback and BH accretion are calibrated based on present-day galaxy 
stellar mass function while also requiring reasonable galaxy sizes.
Additionally the AGN feedback efficiency is constrained by the central 
black hole-galaxy mass relation. The simulations resolve galaxies over 
the range $10^{8} < M_{\star}/M_{\sun} \lesssim 10^{11}$. % and reproduce the galaxy  stellar mass function to $\lesssim 0.2~\mathrm{dex}$ over this range. 
Of the EAGLE project simulations, we use L0100Ref, which has a volume 
of $(100\,\mathrm{comoving\,Mpc})^3$ and baryonic mass resolution of 
$1.81\times 10^6M_{\sun}$. The SFR--$M_*$ relation and quiescent fractions 
in the EAGLE simulations have been previously discussed 
in~\citet{furlong2015, trayford2015, trayford2017}. 

\subsection{{\sc Mufasa}} \label{sec:mufasa}
{\sc Mufasa} is a hydrodynamic simulation constructed with {\sc Gizmo}, which 
is a code built on {\sc Gadget} that uses the Meshless Finite Mass (MFM) 
hydrodynamics method~\citep{hopkins2015a} rather than SPH.  %{\sc Mufasa} is constructed  using {\sc Gizmo}, a code built  upon {\sc Gadget} where we employ the Meshless Finite Mass (MFM) hydrodynamics  method~\citep{hopkins2015a} rather than SPH.
{\sc Mufasa} includes star formation via a Kennicutt-Schmidt law based on the 
molecular hydrogen density as computed using the sub-grid recipe in~\citep{krumholz2011}, 
two-phase kinetic outflows with scalings as predicted in the Feedback in 
Realistic Environments (FIRE) simulations~\citep{muratov2015}, and quenching 
in massive galaxies by keeping all non-self shielded gas within halos above 
a mass of $M_q>(1+0.48z)10^{12}M_\odot$~\citep{mitra2015} near the halos' 
virial temperature~\citep{gabor2015}. {\sc Mufasa} has a box size of 
$50\,h^{-1}\ {\rm Mpc}$ and particle masses of $9.6 \times 10^7\ M_{\odot}$ 
and $1.82 \times 10^7\ M_{\odot}$ for dark matter and baryons, respectively. 
The stellar mass function evolution, gas and metal content of galaxies, and 
color-mass diagram of {\sc Mufasa} have been previously discussed in~\citet{dave2016,dave2017,dave2017b}]
%These prescriptions yield predictions  that are in good agreement with a range of observations across cosmic time, including the galaxy stellar mass function evolution~\citep{dave2016}, the  gas and metal content of galaxies~\citep{dave2017b} and the color-mass diagram~\citep{dave2017}. It has some difficulties in over-quenching satellite galaxies in massive halos~\citep{rafieferantsoa2018} and likely producing  too high X-ray luminosities~(Robson et al., in prep.). 
%Modulo these caveats, {\sc Mufasa} represents a state of the art model for studying galaxy assembly, particularly for central galaxies.

\subsection{the Santa-Cruz Semi-Analytic Model} \label{sec:scsam}
The `Santa Cruz' SAM (SC-SAM) is a semi-analytic model run on 
merger trees from the Bolshoi--Planck dark matter only $N$-body 
simulations~\citep{rodriguez-puebla2016}. The model includes 
schematic prescriptions for gas heating and cooling, multi-phase 
gas partitioning, star formation, chemical evolution, feedback 
from stars, supernovae and SMBHs, the sizes of galactic disks and 
bulges, and merger-induced starbursts and structural transformations. 
The SC-SAM was first presented in~\cite{somerville1999} and 
\cite{somerville2001}, with significant updates described 
in~\cite{somerville2008, somerville2008a, somerville2012, porter2014, 
popping2014, somerville2015a}. In this work, we use the version of 
the SC-SAM described in~\cite{popping2014} and~\cite{somerville2015a}, 
which includes recipes for partitioning multi-phase gas into H I, 
H$_2$ and H II (specifically we adopt the~\citealt{gnedin2011} 
partitioning recipe). The Bolshoi--Planck simulations have particle 
masses of $1.5 \times 10^8 M_\odot$. Based on this resolution limit, 
we focus our analysis on halos with $M_h > 10^{11} M_\odot$.
Since this roughly corresponds to $M_* \sim 10^{8.5} M_\odot$ at 
$z\sim 0$, we impose a conservative $M_* > 10^{8.8}M_\odot$ cut 
to our SC-SAM galaxy sample. We use a ($100$ comoving Mpc/$h$)$^3$
sub-volume for this paper. The properties of the SC-SAM galaxy population, 
such as the quiescent fraction have been previously discussed 
in~\cite{brennan2015,somerville2015b,brennan2017,pandya2017}.

%We run the SC-SAM on merger  trees extracted directly from the Bolshoi--Planck dark matter-only $N$-body simulations~\citep{rodriguez-puebla2016}, using a subvolume spanning  ($100$ comoving Mpc/$h$)$^3$ for this paper. 
%Previous studies have shown that the SC-SAM reproduces well many properties  of the galaxy population both nearby and out to moderately  high-redshift~\citep[\emph{e.g.}][]{somerville2015b}. In particular,  \cite{brennan2015,brennan2017,pandya2017} showed that the SC-SAM reproduces  the observed quenched fraction at $z\sim 0$ but significantly underestimates  it at $z\sim 2$. 

\subsection{Observed SDSS Galaxies} \label{sec:obvs}
Our main focus in this work is to compare galaxies from our 
simulations. However, since the goal of simulations is to 
reproduce observations, we include, for reference, galaxies 
from the SDSS volume-limited sample and the NSA low-luminosity 
galaxy sample. We provide a brief description of the two 
observed galaxy samples below. 

For the SDSS sample, we follow the sample selection of~\cite{tinker2011}. 
We construct the SDSS volume-limited galaxy sample with $M_r - 5\log(h) < -18$ 
and complete in $M_* > 10^{9.7} M_\odot$ from the NYU Value-Added 
Galaxy Catalog \citep[VAGC;][]{blanton2005} which corresponds to 
the SDSS Data Release 7~\citep[DR7;][]{abazajian2009} at redshift 
$z \approx 0.04$. For further details on the sample, we refer readers 
to~\cite{tinker2011,wetzel2013,hahn2017}.

At lower stellar masses, we use an isolated dwarf galaxy sample 
selected from the NSA catalog as described in \citet{geha2012}. 
Briefly, the NSA catalog is a reprocessing of SDSS DR8, optimized 
for low-luminosity objects. It relies on the improved background 
subtraction technique of \cite{blanton2011}. The catalog extends 
to $z \approx 0.055$ and includes re-calibrated spectroscopy~\citep{yan2011,yan2012} 
with much smaller errors. However, this recalibration is mostly 
relevant only at small equivalent width values and hence does not 
largely affect galaxies on the SFS. %Dwarf galaxies are considered  isolated, and selected, when the distance from a more massive host  is $> 1.5 {\rm Mpc}$ \citep{geha2012}.

For both galaxy samples, the stellar masses are estimated using the 
\citet{blanton2007} $\mathtt{kcorrect}$ code, which assumes a 
\cite{chabrier2003} IMF. The SFRs are from the current release 
of~\citet{brinchmann2004}\footnote{http://www.mpa-garching.mpg.de/SDSS/DR7/}, 
where they are derived using the~\cite{bruzuala.1993} model with the 
\cite{charlot2000} dust prescription and CLOUDY \citep[version C90.04;][]{ferland1996} emission line modeling.  
For galaxies classified as having an AGN or a composite spectrum, the 
SFR is measured from the $D_n4000$ index~\citep{balogh1998}. 
Additionally, for star-forming galaxies that have low S/N spectra, the SFR 
is derived from the $H{\alpha}$ luminosity~\citep{brinchmann2004}. 
We emphasize that SSFRs $\lesssim 10^{-12} \mathrm{yr}^{-1}$ should only be 
considered upper limits to the true value~\citep{salim2007}.

%are derived following a Bayesian framework where all emission lines are modeled based on the \citet{bruzualCharlot1993, bruzualCharlot1993} galaxy evolution models combined with a dust prescription following \citet{charlotFall2000} and emission line modeling using CLOUDY \citep[version C90.04][]{ferland1996} as described in \citet{charlotLonghetti2001, charlot2002}. 
%For galaxies that are classified as containing an AGN or have a composite (SF and AGN) spectrum, the relation between the derived SFR and Dn4000 index \citep[implemented following][]{Balogh1998} is used to measure the SFR from the Dn4000 index value. Additionally, for galaxies that are starforming but have low S/N spectra the SFR is only dependent on the $H{\alpha}$ luminosity, where the conversion factor used is taken from the distribution function of conversion factors for starforming galaxies with high S/N in the same stellar mass bin \citep{brinchmann2004}. 

%%%%%%%%%%%%%%%%%%%%%%%%%%%%%%%%%%%%%%%%%%%%%%%%%%%%
% Section 
%%%%%%%%%%%%%%%%%%%%%%%%%%%%%%%%%%%%%%%%%%%%%%%%%%%%
\subsection{Identifying Isolated/Central Galaxies} \label{sec:central}
%Galaxies carry the imprint of their environment (~\citealt{hubble1936, oemler1974, dressler1980, guzzo1997},  for a recent review see~\citealp{blanton2009}). 
As quiescent fraction~\citep[\emph{e.g.}][]{baldry2006,peng2010,hahn2015}
and star formation quenching timescale~\citep{wetzel2013,hahn2017} 
measurements suggest, depending on whether a galaxy is a central or satellite, 
its star formation is subject to different physical mechanisms. There may 
also be significant discrepancies between the SFSs of central versus
satellite galaxies~\citep{wang2018}. Without delving further into the 
environment dependence, in this paper \emph{we focus solely on the 
central galaxies, which constitute the majority of massive galaxies 
($M_* > 10^{9.5}M_\odot$) at $z \sim 0$}. 
%Galaxies and their detailed properties carry the  imprint of their environment (~\citealt{hubble1936, oemler1974, dressler1980, guzzo1997},  for a recent review see~\citealp{blanton2009}). The environment dependence of, for instance, the quiescent  fraction~\citep[\emph{e.g.}][]{baldry2006,peng2010,hahn2015} and the timescale of star formation quenching~\citep{wetzel2013,hahn2017},  suggest that different physical mechanisms act in different environments  to impact galaxy star formation. 

Galaxy environment (centrals/satellites), despite its importance, 
is often heterogeneously defined in the literature~\cite{muldrew2012}. 
Even among our simulations, the classification of centrals depends on 
the definition of halo properties, and thus on the underlying halo 
finders. EAGLE and Illustris use $\mathtt{SUBFIND}$~\citep{springel2001},
where halos are defined as locally overdense, gravitationally bound
(sub)structures within a connected region selected through a 
friend-of-friends~\citep[FOF;][]{davis1985} group finder. {\sc Mufasa} 
and the SC-SAM, meanwhile, are based on halos found using 
$\mathtt{ROCKSTAR}$~\citep{behroozi2013}, which defines halos using a
hierarchical phase-space based FOF technique and seeks to maximize 
the consistency of the halo through time. In addition, these central 
classifications also use information of the underlying dark 
matter --- information {\em not} available in observations. Therefore, 
we identify central galaxies in all of our simulations consistently 
using %an extended version of 
the~\cite{tinker2011} group finder, 
designed to identify satellite/centrals in observations. 
%Despite the consistent definition of centrals in the simulations we use, the classification depends on the definition of halo properties in the simulation and thus differences can arise due  to differences in the underlying halo finders used. In the case of our set of simulations both EAGLE and Illustris use
%SUBFIND~\citep{springel2001}, where halos are defined as locally overdense, gravitationally bound (sub)structures within a connected region selected through a friend-of-friends~\citep[FOF]{davis1985} group finder. On the other hand {\sc Mufasa} and the Santa Cruz SAM are based on halos found using ROCKSTAR~\citep{behroozi2013}, which defines halos using a hierarchical phase-space based FOF technique and endeavors to maximize the consistency of the halo through time. Comparison projects have shown that the halo (central)/subhalo (satellite) classification generally agrees well (although the (sub)halo mass estimation can vary significantly), except when the subhalo is close to the center of the host for example during major mergers, and in late stages of minor mergers~\citep{knebe2011, behroozi2015}. 

The~\cite{tinker2011} group finder is a halo-based algorithm that uses 
the abundance matching ansatz to iteratively assign halo masses to groups. 
It assigns a tentative halo mass to each galaxy by matching the abundance 
of the objects. Then starting with the most massive galaxy, nearby lower
mass galaxies are assigned a probability of being a satellite. Once all 
the galaxies are assigned to a group, the halo masses of the central galaxies 
are updated by abundance matching with the total stellar mass in the groups. 
This entire process is repeated until convergence. In the resulting catalog, 
every group contains one central galaxy, which by definition is the 
most massive, and a group can contain zero, one, or many satellites.
For a detailed description we refer readers to~\cite{tinker2011,wetzel2012}. 

Overall, we find good agreement between the central classifications of 
the group finder and simulations. Using purity and completeness as defined 
in Eqs.~13 and~15 of~\cite{campbell2015}, we find central galaxy classification purities of 
$99\%, 93\%, 84\%$, and $90\%$ and completenesses of $86\%$, $89\%$, 
$91\%$, and $86\%$  for the Illustris, EAGLE, {\sc Mufasa} and SC-SAM 
simulations respectively. Differences in the purity and completeness for the 
simulations is likely due to the different halo finders used in the 
simulations as described above. For the hydrodynamic simulations, we find no 
significant stellar mass dependence in the purities. The SC-SAM, due to 
its mass resolution, has purity of $44\%$ at $M_*$ below $10^{8.5}M_\odot$ 
and $97\%$ above. However, as we discuss in Section~\ref{sec:scsam}, we 
impose a conservative stellar mass limit of $M_* > 10^{8.8} M_\odot$. 
As expected from the high purity and completeness, the group finder 
classification does \emph{not} impact the results of this paper. In 
the next section, we proceed to fitting the star formation sequence 
of simulated galaxies identified as centrals.


%%%%%%%%%%%%%%%%%%%%%%%%%%%%%%%%%%%%%%%%%%
% Figure 
%%%%%%%%%%%%%%%%%%%%%%%%%%%%%%%%%%%%%%%%%%
\begin{figure*}
\begin{center}
\includegraphics[width = 0.9\textwidth]{figs/SFMSfit_demo.pdf} 
\caption{
We illustrate our GMM SFS fitting method for Illustris central galaxies in two 
stellar mass bins highlighted on the SFR--$M_*$ relation of the left panel: 
$10.4 < \log\,M_* < 10.6$ and $11.0 < \log\,M_* < 11.2$. On the right, we compare the SSFR 
distributions, $p(\log\,\mathrm{SSFR})$, in the two stellar 
mass bins to their best-fit GMMs. The $p(\log\,\mathrm{SSFR})$ in the center panel is best described by a 
GMM with three components (orange, green, and blue) while the
$p(\log\,\mathrm{SSFR})$ in the right panel is best described by 
a GMM with two components (orange and blue). The SFS components of the 
best-fit GMMs are plotted in blue. \emph{Our GMM SFS fitting provides
a flexible and data-driven way to identify the SFS for a wide variety 
of SSFR distributions without hard assumptions or cuts to the sample.}
}\label{fig:fitdemo}
\end{center}
\end{figure*}
%%%%%%%%%%%%%%%%%%%%%%%%%%%%%%%%%%%%%%%%%%

\section{Fitting the Star Forming Sequence}\label{sec:sfmsfit}
We present the SFR--$M_*$ relation of central galaxies from the 
simulations and observations of Section~\ref{sec:galsims} in 
Figure~\ref{fig:sfrmstar}. Regardless of SFR timescale (top/bottom),
for both simulations and observations, and over four orders of magnitude 
in stellar mass, \emph{the SFR and $M_*$ of star-forming galaxies lie 
tightly correlated on the SFS}. A well-defined SFS can be found in 
each panel of Figure~\ref{fig:sfrmstar}.

Despite its universality among observations and simulations, 
different datasets give rise to significantly different SFR--$M_*$ 
distributions. This makes the SFS difficult to \emph{consistently} 
quantify. So far in the literature, a wide variety of fitting 
methods have been applied to data --- even in the same comparison 
(see Appendix~\ref{app:literature}). \cite{bluck2016} fit the SFS 
using median $\log\mathrm{SFR}$s of galaxies with $M_* < 10^{10}M_\odot$ 
and extrapolate to higher masses. This method, however, assumes that 
all $M_* < 10^{10}M_\odot$ galaxies lie on the SFS and that there 
is no variation in the slope of the SFS at higher stellar masses. 
Alternatively, \cite{lee2015} fit the SFS using median 
$\log\,\mathrm{SFR}$s of galaxies in the sample after some color-color 
cut to identify SF galaxies. Other recent works in the literature have opted 
for more sophisticated methods such as fitting a three-component
Gaussian~\citep{bisigello2018} or a zero-inflated negative binomial 
distribution~\citep{feldmann2017}. 

All of these methods require arbitrary assumptions or hard cuts on the 
sample. Moreover, these methods struggle to flexibly account for 
the different features in the galaxy property space over a wide 
SFR or $M_*$ range and in different simulations and observations. %for the wide variety of SSFR distributions we see  in simulations and observations. 
Even for fixed a stellar mass bin ($10.4 < \log M_* < 10.6$), 
Figure~\ref{fig:pssfr} reveals the significant discrepancies in 
the SSFR distributions of our four simulations. In an effort to 
better fit a wide variety of SFR--$M_*$ distributions and to relax the assumptions and cuts imposed 
on the data, \emph{we present a flexible and data-driven method for 
fitting the SFS that makes use of Gaussian Mixture Models}.

\subsection{Using Gaussian Mixture Models}
Gaussian mixture models (hereafter GMM), and mixture models in general, provide 
a probabilistic way of describing the distribution of a population by 
identifying subpopulations from the data~\citep[][]{Press:1992:NRC:148286, 9780471006268}.
Besides their extensive use in machine learning and statistics, 
GMMs have also been used in wide range of astronomical analyses~\citep[\emph{e.g.}][]{bovy2011,lee2012,taylor2015}. 
Since identifying the subpopulation of star forming galaxies from the overall
galaxy population is equivalent to fitting the SFS, GMMs provides a 
well-motivated, data-driven, and effective method to tackle the problem. 

A GMM, more precisely, is a weighted sum of $k$ Gaussian component densities 
\begin{equation} \label{eq:gmm}
\hat{p}(x;\bm{\theta}) = \sum\limits_{i=1}^{k} \pi_i \, \mathcal{N}(x; \bm{\theta}_i),
\end{equation}
which can be used to estimate the density. The weights, $\pi_i$, mean, and 
variance $\bm{\theta}_i=\{\mu_i, \sigma_i\}$ of the components are free 
parameters in the GMM. For a given data set $\{x_1, ..., x_n\}$, these 
parameters are most commonly estimated through the expectation-maximization 
algorithm~\citep[EM;]{dempster1977,neal1998}. 

Starting with randomly assigned $\bm{\theta}_{i}^0$ to the $k$ GMM components, 
the EM algorithm iterates between two steps. First, for every data point, 
$x_i$, the algorithm computes for a probability of $x_i$ being generated by 
each GMM component. These probabilities act as assignment weights to each of
the components. Next, based on these weights, $\bm{\theta}_i^t$ of the components 
are updated to $\bm{\theta}_i^{t+1}$ to maximize the likelihood of the assigned 
data. $\pi_i$ are also updated by summing up the assignment weights and 
normalizing the sum by the total number of data points. These steps are 
repeated until convergence --- \emph{i.e.} when $p(\{x_1, ..., x_n\} ; \bm{\theta}_t)$ 
converges. Instead of starting with randomly assigning $\bm{\theta}_{i}^0$, 
we initiate our EM algorithm using a $k$-means clustering algorithm~\citep{lloyd1982},
more specifically we use the $k$-$\mathtt{means}$++ algorithm~\citep{arthur2007}. 

For our actual SFS fitting method, we first divide the galaxy 
sample into stellar mass bins of width $\Delta \log\,M∗$. In this paper 
we use bins of $\Delta \log\,M∗ = 0.2\ \mathrm{dex}$; however, this 
choice does not significantly impact the final fit. For each stellar 
mass bin, if there are more than $N_\mathrm{thresh}{=}100$ galaxies in the bin, 
we fit the SSFR distribution using GMMs with $k{=}1$ to 3 components with 
parameters determined from the EM algorithm described above. 
Our restriction to models with a maximum of 3 components is 
motivated by the three main galaxy subpopulations: quiescent, star-forming, 
and transitioning galaxies. More importantly, for our observed galaxy samples
and hydrodynamic simulations, even when we allow for more than 3 components,
the best-fit GMMs have $k\leq3$. Hence, the choice of $k\leq3$ does not 
significantly the results of this work.
%Our restriction  to models with a maximum of 3 components is motivated by the three main  galaxy classifications: quiescent, star-forming, and transitioning  populations. Furthermore, for our observed galaxy samples, even when we allow for more than 3 components,  the best-fit GMMs have $k\leq3$. We confirm that restricting ourselves  to 3 components does not significantly impact the results of this work (Appendix~\ref{app:gmm}). 

Out of the three ($k{\leq}3$) GMMs, we select the one with the lowest Bayesian 
Information Criteria~\citep[BIC;][]{schwarz1978} as our ``best-fit'' model. 
BIC is often used in conjunction with GMMs~\citep[\emph{e.g.}][]{leroux1992,roeder1997,fraley1998,steele2010performance} 
and also more generally for model selection in 
astronomy~\citep[\emph{e.g.}][]{liddle2007,broderick2011,vakili2016}.
In addition to the likelihood, BIC introduces a penalty term for the number
of parameters in the model. This way, using BIC not only finds a good fit to 
the data, but it also addresses the concern of over-fitting. 
Next, we designate the component of the best-fit GMM with mean 
$\log\,\mathrm{SSFR} > −11$ as the SFS component. If there are more 
than one GMM component with mean $\log\,\mathrm{SSFR} > −11$, 
we take the component with the larger weight (\emph{i.e.} the mode) 
as the SFS component. 
{\color{red} In Appendix~\ref{app:gmm_pssfr}, we present a detailed 
comparison of the GMM fits to the SSFR distributions of our simulations 
and discuss the advantages of our SFS fitting method in further detail.} 

%When we compare  the $k\leq3$ GMM fits to the SSFR distribution of our simulations in three  stellar mass bins, we find good agreement between the lowest BIC best-fit  models and the SSFR distributions (Appendix~\ref{app:gmm_pssfr};  Figures~\ref{fig:pssfr_gmm_inst} and~\ref{fig:pssfr_gmm_100myr}).

In Figure~\ref{fig:fitdemo}, we illustrate our GMM SFS fitting for the 
central galaxies of the Illustris simulation in two stellar mass 
ranges highlighted in the left panel: $10.4 < \log\,M_* < 10.6$ (center) 
and $11.0 < \log\,M_* < 11.2$ (right). For the two stellar mass bins, 
we compare the SSFR distributions of the bins to the components of the 
best-fit GMMs derived from our SFS fitting method. The SFS component 
is plotted in blue. The SSFR distribution of the center panel is best 
described by a GMM with three components while the SSFR distribution 
in the right  panel is best described by a GMM with only two components.
%The right panels illustrate the wide variation in SSFR distributions of the galaxy samples. 
These comparisons highlights the flexibility and effectiveness 
of our fitting method in identifying the SFS for different SSFR 
distributions. 
All the code used for our SFS fitting is publicly available 
at \url{https://github.com/changhoonhahn/LetsTalkAboutQuench}.

%%%%%%%%%%%%%%%%%%%%%%%%%%%%%%%%%%%%%%%%%%
% Figure
%%%%%%%%%%%%%%%%%%%%%%%%%%%%%%%%%%%%%%%%%%
\begin{figure*}
\begin{center}
\includegraphics[width = 0.8\textwidth]{figs/Catalogs_SFMSfit_SFRinst.pdf} 
\caption{Best-fit SFS of the central galaxies in the Illustris, EAGLE, {\sc Mufasa}, 
    and SC-SAM simulations as identified by our SFS GMM fitting method 
    (Section~\ref{sec:sfmsfit}). The SFSs above are fit from the instantaneous 
    SFR--$M_*$ relation. For reference, we include the best-fit SFS of the SDSS 
    sample in the top right panel. The uncertainties of the best-fit SFS are
    derived using bootstrap resampling. When we compare the SFS fits (bottom 
    right panel), we find that \emph{the SFSs of the simulations have similar 
    overall stellar mass dependence, but their amplitude vary roughly by an 
    order of magnitude.}} \label{fig:sfmsfit_inst}
\end{center}
\end{figure*}
%%%%%%%%%%%%%%%%%%%%%%%%%%%%%%%%%%%%%%%%%%

%%%%%%%%%%%%%%%%%%%%%%%%%%%%%%%%%%%%%%%%%%
% Figure  
%%%%%%%%%%%%%%%%%%%%%%%%%%%%%%%%%%%%%%%%%%
\begin{figure*}
\begin{center}
\includegraphics[width = 0.8\textwidth]{figs/Catalogs_SFMSfit_SFR100myr.pdf} 
    \caption{Same as Figure~\ref{fig:sfmsfit_inst} but for $100\,\mathrm{Myr}$ SFR. 
    As in Figure~\ref{fig:sfmsfit_inst}, \emph{the SFSs of the simulations have 
    similar stellar mass dependence but vary roughly by an order of magnitude in amplitude.}}
\label{fig:sfmsfit_100myr}
\end{center}
\end{figure*}
%%%%%%%%%%%%%%%%%%%%%%%%%%%%%%%%%%%%%%%%%%

%%%%%%%%%%%%%%%%%%%%%%%%%%%%%%%%%%%%%%%%%%
% Figure  
%%%%%%%%%%%%%%%%%%%%%%%%%%%%%%%%%%%%%%%%%%
\begin{figure}
\begin{center}
\includegraphics[width = 0.48\textwidth]{figs/Catalogs_SFMS_powerlawfit.pdf} 
\caption{Power-law fits to the best-fit SFS for the Illustris (green), 
    EAGLE (red), {\sc Mufasa} (purple), and SC-SAM (brown) simulations. 
    We use instantaneous SFR and $100\,\mathrm{Myr}$ SFR in the left
    and right panels respectively. We list the best-fit parameters 
    in Table~\ref{tab:sfms_powerlaw}. For a more consistent 
    comparison, we restrict the power-law fit to the {\sc Mufasa} SFS 
    to $\log\,M_* < 10.6$ (purple solid) since the fit is 
    significantly impacted by the high stellar mass turnover (purple dotted). 
    The power-law fits further illustrate the ${\sim}1\,\mathrm{dex}$ 
    discrepancy among the SFSs of the simulations.}
    \label{fig:sfmsfit_powerlaw}
\end{center}
\end{figure}
%%%%%%%%%%%%%%%%%%%%%%%%%%%%%%%%%%%%%%%%%%

\section{Results} \label{sec:results}
\subsection{SFS of simulated galaxies} \label{sec:sfs}
Now using our SFS fitting method from above, % for fitting the SFS, which can be flexibly  applied a wide range of SSFR distributions, 
we can identify the SFSs of our simulated central galaxies from 
Section~\ref{sec:galsims}. We present the best-fit SFSs of our
simulated galaxies from the Illustris, EAGLE, {\sc Mufasa}, and SC-SAM 
simulations for the instantaneous and $100\,\mathrm{Myr}$ SFR 
timescales in Figures~\ref{fig:sfmsfit_inst} and~\ref{fig:sfmsfit_100myr}, 
respectively. %For reference we include the best-fit SFS of the SDSS central galaxies in the top right panel. 
Overall, the best-fit SFSs of the simulations exhibit a similar monotonic
relation between SFR and $M_*$ out to $M_* \gtrsim 10^{11}M_\odot$, for 
both SFR timescales. However, when we compare the best-fit SFSs in more
detail (bottom right panels of Figures~\ref{fig:sfmsfit_inst} 
and~\ref{fig:sfmsfit_100myr}), we find \emph{order of magnitude discrepancies 
in SFR among the best-fit SFSs throughout the stellar mass range of the 
simulations for both SFR timescales}. 

%The best-fit SFSs of the Illustris, EAGLE, and SC-SAM simulations
%exhibit a similar monotonic relation between SFR and $M_*$ out to 
%$M_* \gtrsim 10^{11}M_\odot$, for both SFR timescales. 
%Between the two SFR timescales, we find little difference 
%between the best-fit SFSs for each simulation. However, when we compare the 
%best-fit SFSs of the simulations altogether in more detail, we find 
%These best-fit SFSs allow us to make detailed comparison among the simulations. 
The uncertainties for the best-fit SFSs in Figures~\ref{fig:sfmsfit_inst} 
and~\ref{fig:sfmsfit_100myr} are derived from bootstrap resampling~\citep{efron1979} 
in each stellar mass bin of the fitting. These uncertainties are 
\emph{underestimates} because they do not account for 
``cosmic variance'' --- \emph{i.e.} we only have one finite volume realization 
of each simulation. Furthermore, uncertainty of the best-fit SFS corresponds 
to the uncertainty of the means of the SFS GMM component, which is only one 
of the parameters in the GMM. Uncertainties from bootstrap resampling, 
however, do \emph{not} account for the correlations between the mean of 
the SFS GMM and other parameters of the GMM in Eq.~\ref{eq:gmm}. A more 
robust estimate of the uncertainties would involve estimating the marginalized 
posterior distribution of the SFS GMM component mean using a method like MCMC. 
Since this still does not account for cosmic variance, we use bootstrap 
uncertainties. 

Given the best-fit SFSs, we can further parameterize the SFS to some 
functional form as often done in the literature --- \emph{e.g.} 
power-law~\citep{speagle2014} or broken power-law~\citep{lee2015}. With 
little evidence of a turnover in the SFS in {\em most} of simulations, 
we fit a power-law of the form 
\begin{equation} \label{eq:powerlaw}
\log\,\mathrm{SFR}_\mathrm{MS} = m\,(\log\,M_* - 10.5) + b
\end{equation}
to the SFSs in Figure~\ref{fig:sfmsfit_powerlaw}. Unlike the SFS of 
other simulations, the best-fit SFSs for {\sc Mufasa} have a significant 
turnover at $M_*{\sim}10^{10.5}M_\odot$. This turnover is \emph{not} 
caused by  misidentification of the SFS or some systematic effect of the 
fitting. {\color{red} Instead, the turnover is due to the halo mass 
dependent quenching prescription in {\sc Mufasa} (Section~\ref{sec:mufasa}), 
which causes a sharp cut-off in the SFS, in comparison to the other more 
self-consistent AGN feedback models. Since we're primarily interested in
the SFS, we fit Eq.~\ref{eq:powerlaw} to {\sc Mufasa}'s SFS over the range 
$M_*{<}10^{10.6} M_\odot$, below the turnover.} 

Comparing the best-fit (maximum likelihood) parameters of Eq.~\ref{eq:powerlaw}
that we list in Table~\ref{tab:sfms_powerlaw},  we find significant
%The power-law parameterizations accentuate the stellar mass dependence of  the SFSs and can reveal the mass dependence in the SFS discrepancies. 
discrepancies in the slopes of the SFSs of the simulations with SC-SAM 
having the steepest SFS with $m > 1$. Furthermore, we also find significant
discrepancies ($> 0.5\,\mathrm{dex}$) in the amplitudes of the SFSs at 
$M_* = 10^{10.5} M_\odot$ ($b$ in Eq.~\ref{eq:powerlaw}). Although the power-law
SFS fits highlight the stellar mass dependence of the SFSs, they reveal  
no consistent stellar mass dependence in the discrepancies of the SFSs. For 
both SFR timescales, the discrepancies among the SFSs range from $0.5 - 1.\,\mathrm{dex}$ 
throughout the stellar mass range (Figure~\ref{fig:sfmsfit_powerlaw}). 

%The discrepancies between the instantaneous SFR SFSs of Illustris, EAGLE, and SC-SAM range from $1.\,\mathrm{dex}$ at low $M_*$ to $0.5\,\mathrm{dex}$ at high $M_*$ . Meanwhile, 
%for $100\,\mathrm{Myr}$ SFR, we find slightly greater discrepancies among Illustris, EAGLE, and SC-SAM throughout the stellar mass range.
%\begin{itemize}
%\item \todo{What's causing the simulations to have SFSs that are almost an 
%order of magnitude different? -> Include reference to appendix \ref{app:literature}} 
%\end{itemize}
Despite the order of magnitude discrepancies among the SFRs of the SFSs,
we find more consistent cosmic star formation densities among the 
simulations. For Illustris, EAGLE, {\sc Mufasa}, and SC-SAM, respectively, 
we estimate cosmic star formation densities of 
$-1.66, -2.22, -1.87$, and $-1.94\,M_\odot yr^{-1} \mathrm{Mpc}^{-3}$ 
using instantaneous SFRs and similarly 
$-1.68, -2.20, -1.91$, and $-1.94$ $M_\odot yr^{-1} \mathrm{Mpc}^{-3}$ 
using $100\,\mathrm{Myr}$ SFRs. In fact, these star formation densities
are more or less consistent with observations~\citep{madau2014}. 
{\color{red}
The cosmic star formation densities roughly correspond to the total 
star formation in the SFS weighted by the stellar mass function (SMF).
The SMFs of our simulations are less discrepant than the SFSs; as
a result, the cosmic SF densities have smaller discrepancies.
For example, {\sc Mufasa} has the highest SFS for most of the 
$M_*$ range in Figures~\ref{fig:sfmsfit_inst} and~\ref{fig:sfmsfit_100myr}), 
however, its SMF has an overall lower amplitude than Illustris and 
SC-SAM. Hence it has a SF density similar to SC-SAM and lower than 
Illustris.}

In addition to its position, $\mu_\mathrm{SFS}$, the SFS GMM component 
is also described by $\sigma_\mathrm{SFS}$ --- the width of the SFS. 
Using $\sigma_\mathrm{SFS}$ derived from the SFS GMM fitting, 
we can compare the width of the SFS among the simulations 
(Figure~\ref{fig:sfms_width}). The uncertainties for the widths are 
calculated through bootstrap resampling in the same way as the 
uncertainties for the SFS fits. Overall, we find little stellar mass 
dependence in $\sigma_\mathrm{SFS}$ for the simulations. For Illustris, 
EAGLE, {\sc Mufasa}, and SC-SAM we respectively find 
$\sigma_\mathrm{SFS} \sim 0.2, 0.26, 0.25$, and $0.24\,\mathrm{dex}$ 
for instantaneous SFR and
$\sigma_\mathrm{SFS} \sim 0.18, 0.19, 0.27$, and $0.23\,\mathrm{dex}$
for $100\,\mathrm{Myr}$ SFR. These $\sigma_\mathrm{SFS}$ are narrower 
than the ${\sim}0.3\,\mathrm{dex}$ width measured in 
observations~\citep[\emph{e.g.}][]{daddi2007, noeske2007, magdis2012, whitaker2012}. 
This difference, however, does not account for the uncertainties in the 
observed SFR measurements, which would reduce the discrepancy. We 
therefore conclude that \emph{the width of the SFS from our 
simulations are in agreement with the observed SFS width.}

%%%%%%%%%%%%%%%%%%%%%%%%%%%%%%%%%%%%%%%%%%
% Table
%%%%%%%%%%%%%%%%%%%%%%%%%%%%%%%%%%%%%%%%%%
\begin{table}
\caption{Power-law fit to the SFS of our simulated central galaxies from the
Illustris, EAGLE, {\sc Mufasa}, and SC-SAM simulations.} 
\begin{center}
\begin{tabular}{p{4cm}cc} \toprule
\multicolumn{3}{c}{Star Forming Sequence power-law fit} \\ [3pt]
\multicolumn{3}{c}{$\log\,\mathrm{SFR}_\mathrm{MS} = m\,(\log\,M_* - 10.5) + b$  } \\ [3pt]
Simulation & $m$ & $b$ \\ 
\hline
\multicolumn{3}{c}{Instantaneous SFR} \\
Illustris 	& 1.01 & 0.59 \\ 
EAGLE 		& 0.92 & 0.24 \\ 
{\sc Mufasa} 		& 0.76 & 0.59 \\ 
{\sc Mufasa}$^*$ 	& 0.89 & 0.73 \\ 
SC-SAM 		& 1.17 & 0.48 \\ 
\hline \hline
\multicolumn{3}{c}{$100\,\mathrm{Myr}$ SFR} \\
Illustris 	& 0.93 & 0.52 \\
EAGLE  		& 0.75 & 0.21 \\
{\sc Mufasa}		& 0.43 & 0.36 \\
{\sc Mufasa}$^*$ 	& 0.99 & 0.84 \\ 
SC-SAM 		& 1.16 & 0.47 \\ 
\hline
\end{tabular} \label{tab:sfms_powerlaw}
\end{center}
$^*$ power-law fit to the {\sc Mufasa} SFS below its turnover ($\log\,M_* {<}\,10.6$)
\end{table}
%%%%%%%%%%%%%%%%%%%%%%%%%%%%%%%%%%%%%%%%%%

One factor that impacts our SFS fits is the strict lower limit of the 
$\log\mathrm{SFR}$s caused by the resolution effects in the simulations. 
This is particularly evident in the $100\,\mathrm{Myr}$ SFR--$M_*$ 
relations of the hydrodynamic simulations of Figure~\ref{fig:sfrmstar}---especially 
{\sc Mufasa}. As we describe in Section~\ref{sec:galsims}, the $100\,\mathrm{Myr}$ 
SFRs are  calculated using the ages of all star particles in a galaxy. For a galaxy to 
have star formation (\emph{i.e.} SFR $> 0$), it must \emph{at least} 
form one star particle over the last $100\,\mathrm{Myr}$. A single star particle 
forming over $100\,\mathrm{Myr}$ amounts to a SFR of 
${\sim}0.02\ M_{\sun}\ yr^{-1}$ for Illustris and EAGLE and
${\sim}0.2\ M_{\sun}\ yr^{-1}$ for {\sc Mufasa}. This resolution limit, ultimately 
impacts the SFS fits at stellar masses below $10^{8.4}$, $10^{8.4}$, and 
$10^{9.2}M_\odot$ for Illustris, EAGLE, and {\sc Mufasa} respectively. 
These stellar mass limits have accordingly been imposed on the SFS fits
and power-law SFS fits in Figures~\ref{fig:sfmsfit_100myr} 
and~\ref{fig:sfmsfit_powerlaw}. In Appendix~\ref{app:zerosfr}, we describe
how we derive these stellar mass limits. 

By using the SFS fitting method we present in this paper, we're able to 
conduct a principled data-driven comparison of the SFSs of central galaxies 
from the Illustris, EAGLE, {\sc Mufasa}, and SC-SAM simulations. From 
this comparison, we find that the amplitudes of the SFSs differ from one 
another by an order of magnitude, with no significant stellar mass dependence. 
Furthermore, despite the discrepancies in their amplitude, the SFSs of 
our simulations have similar widths, consistent with observations. 
%%%%%%%%%%%%%%%%%%%%%%%%%%%%%%%%%%%%%%%%%%
% Figure  
%%%%%%%%%%%%%%%%%%%%%%%%%%%%%%%%%%%%%%%%%%
\begin{figure}
\begin{center}
\includegraphics[width=0.48\textwidth]{figs/Catalogs_SFMS_width.pdf}
\caption{The width of the SFS, $\sigma_\mathrm{SFS}$, for our simulated 
    central galaxies from Illustris, EAGLE, {\sc Mufasa}, and SC-SAM 
    (green, red, purple, and brown respectively). The widths are derived from  
    the GMM SFS fitting and its uncertainties are estimated using bootstrap
    resampling in the same way as the SFS fit uncertainties. 
    $\sigma_\mathrm{SFS}$s in our simulations have little stellar mass 
    dependence and, naively accounting for measurement errors in observed 
    SFR, they are consistent with $\sigma_\mathrm{SFS}{\sim}0.3\,\mathrm{dex}$ 
    from observations (black dashed).} \label{fig:sfms_width}
\end{center}
\end{figure}
%%%%%%%%%%%%%%%%%%%%%%%%%%%%%%%%%%%%%%%%%%

%%%%%%%%%%%%%%%%%%%%%%%%%%%%%%%%%%%%%%%%%%
% Figure  
%%%%%%%%%%%%%%%%%%%%%%%%%%%%%%%%%%%%%%%%%%
\begin{figure*}
\begin{center}
\includegraphics[width=0.95\textwidth]{figs/Catalogs_GMMcomps.pdf} 
\caption{Components of the best-fit GMM from our SFS fitting method for the
    SFR-$M_*$ relations of central galaxies in the Illustris, EAGLE, {\sc Mufasa}, and 
    SC-SAM simulations (left to right). The top and bottom panels use instantaneous 
    SFRs and $100\,\mathrm{Myr}$ SFRs respectively. In each $\log\,M_*$ bin, we mark 
    the SFS component in blue, the lowest SFR component in orange, the intermediate 
    SFR component in green, and the component above the SFS in purple. These components 
    \emph{loosely} correspond to the star-forming, quiescent, transitioning, and 
    star-burst populations. The hydrodynamic simulations, despite the SFS discrepancies,
    (Figures~\ref{fig:sfmsfit_inst} and~\ref{fig:sfmsfit_100myr}), have similar 
    galaxy populations dominated by the SFS and low SF components. Meanwhile in the 
    SC-SAM, the GMM components reveal broad low SFR components that extends out to 
    $\mathrm{SFR} < 10^{-4}M_\odot yr^{-1}$, prominent intermediate components at 
    $M_* \gtrsim 10^{10}M_\odot$, and components above the SFS at $M_* \lesssim 10^{10}M_\odot$.} \label{fig:sfmsfit_comps}
\end{center}
\end{figure*}
%%%%%%%%%%%%%%%%%%%%%%%%%%%%%%%%%%%%%%%%%%

%%%%%%%%%%%%%%%%%%%%%%%%%%%%%%%%%%%%%%%%%%
% Figure 
%%%%%%%%%%%%%%%%%%%%%%%%%%%%%%%%%%%%%%%%%%
\begin{figure*}
\begin{center}
\includegraphics[width=\textwidth]{figs/GMMcomp_composition.pdf} 
\caption{Fractional contributions, $\pi_i$, of the best-fit GMM components from 
    our SFS fitting of the central galaxies in the Illustris, EAGLE, {\sc Mufasa}, 
    and SC-SAM simulations (left to right). We highlight the SFS 
    component in blue, the quenched component in orange, galaxies with 
    SFR$=0$ in red, the transitioning components in green, and the star-burst
    components in purple. For reference, we include $\pi_i$ of the observed 
    centrals from SDSS and NSA in the rightmost panel. In all the simulations, 
    a significant fraction of central galaxies lie below the SFS at 
    $M_* \lesssim 10^9 M_\odot$ in disagreement with observations. For the 
    hydrodynamic simulations, the SFS fraction has little stellar mass dependence
    at $M_* < 10^{11}M_\odot$ unlike observations. Meanwhile, a number of similar 
    components are found between SC-SAM and observations: star-burst 
    components at low $M_*$ and transitioning components at high $M_*$. 
    } \label{fig:kandinsky}
\end{center}
\end{figure*}
%%%%%%%%%%%%%%%%%%%%%%%%%%%%%%%%%%%%%%%%%%

%%%%%%%%%%%%%%%%%%%%%%%%%%%%%%%%%%%%%%%%%%
% Figure 
%%%%%%%%%%%%%%%%%%%%%%%%%%%%%%%%%%%%%%%%%%
\begin{figure*}
\begin{center}
\includegraphics[width=\textwidth]{figs/GMMcomp_comp_uncertainty.pdf} 
\caption{Uncertainties in the fractional contributions, $\pi_i$, of the 
best-fit GMM components from our SFS fitting from Figure~\ref{fig:kandinsky}.
The uncertainties are estimated through bootstrap resampling. 
We plot the SFS component in blue, the quiescent component in orange, 
galaxies with SFR$=0$ in red, the transitioning components in green, and 
the star-burst components in purple. For reference, we include $\pi_i$ of 
the observed centrals from SDSS and NSA in the rightmost panel.
} \label{fig:fcomp_uncertainty}
\end{center}
\end{figure*}
%%%%%%%%%%%%%%%%%%%%%%%%%%%%%%%%%%%%%%%%%%
\subsection{Beyond the SFS of Simulated Galaxies} \label{sec:beyondsfms}
So far we have focused solely on the SFSs of our simulated 
galaxies---\emph{i.e.} 
$\bm{\theta}_\mathrm{SFS} = \{\mu_\mathrm{SFS}, \sigma_\mathrm{SFS} \}$ 
in Eq.~\ref{eq:gmm}. Our GMM fitting method, however, also determines 
$\bm{\theta}_i$ of components other than the SFS. These GMM components 
provide extra features to compare our simulated galaxy samples and also 
offer interesting insights into the different populations in our simulated 
galaxy samples. When we examine $\bm{\theta}_i$ of all components from 
our fitting for our simulated galaxies we find loose correspondence between 
the components and the quiescent, transitioning, and star-burst galaxy 
populations (Figure~\ref{fig:sfmsfit_comps}). To avoid over-interpreting 
this correspondence, we refer to the GMM component with the lowest SFR as 
``low SF'' component, the component with SFR in between the SFS and the 
low SF component as the ``intermediate SF'' component, and finally the 
component with higher SFR than the SFS component as the ``high SF'' component. 
In Figure~\ref{fig:sfmsfit_comps}, we mark the SFS, low SF, intermediate SF, 
and high SF in blue, orange, green, and purple respectively.

Examining the GMM components of the hydrodynamic simulations in 
Figure~\ref{fig:sfmsfit_comps}, we find that a few $\log\,M_*$ bins have 
intermediate SF components in Illustris at $10^9 M_\odot < M_* < 10^{11}M_\odot$.
Also a few of the lowest $\log\,M_*$ bins in Illustris and EAGLE have high SF
components for the $100\,\mathrm{Myr}$ SFRs. Besides these few stellar mass 
bins, however, the central galaxies from the hydrodynamic simulations are
consistently dominated by the SFS and low SF components. Furthermore, 
throughout the stellar mass ranges of the simulations, the low SF 
components in each these simulations have relatively constant widths and 
lie ${\sim}1\,\mathrm{dex}$ below the SFS components.

Unlike the hydrodynamic simulations, however, the low SF components in the 
SC-SAM extend below $\log\,\mathrm{SFR}{=}{-}4$. Furthermore, the intermediate 
and high SF components are much more prominent in the SC-SAM central galaxies. 
At low stellar masses ($M_* \lesssim 10^{10}M_\odot$) every SC-SAM $\log\,M_*$ 
bin has a high SF component. The $\log\,\mathrm{SSFR}$ distributions in these 
bins have extended tails on the higher SFR side of the SFS. Our GMM SFS fitting 
method, thus, fits high SF components in these $\log\,M_*$ bins (bottom left 
and center panels of Figures~\ref{fig:pssfr_gmm_inst} and~\ref{fig:pssfr_gmm_100myr}). 
%However, the differences in BIC values of the best-fit GMM with a high SF  component ($\mathrm{BIC}_{k=3}$; green) versus the GMM without a high SF  component ($\mathrm{BIC}_{k=2}$; orange) in these stellar mass bins are small. 
{\color{red}
These high SF components and the extended range of low SF components are likely 
caused by the re-accretion prescription of the SAM (Section 2.7 of~\citealt{somerville2008a}).
A fraction of gas ejected from halos (\emph{e.g.} from supernovae) is kept in 
a reservoir, which re-collapses into the halos at a later time and becomes available 
again for cooling. The rate of this re-accretion depends on the mass of ejected gas, 
the dynamical time of the halo, and $\chi_\mathrm{re-infall}$---a free parameter 
degenerate with supernovae feedback parameters. This prescription results in bursty 
star formation in the SC-SAM galaxies and causes the extended low SF components and 
the high SF components. %galaxies to have broad range of  SFRs.
}
%The rate of this re-accretion/infall depends on $\chi_\mathrm{re-infall}$, a free parameter, the mass of ejected gas, and the dynamical time of the halo.
% This gas is assumed to accrete on to the halo on a time-scale that is proportional to the halo dynamical time
% The fraction of gas that is ejected from the disc but retained in the halo, versus ejected from the disc and halo, is a function of the halo circular velocity (see S08 for details), such that low-mass haloes lose a larger fraction of their gas. The gas that is ejected from the halo is kept in a larger reservoir, along with the gas that has been prevented from falling in due to the photoionizing background. This gas is assumed to accrete on to the halo on a time-scale that is proportional to the halo dynamical time (see S08 for details)

At high stellar masses ($M_* \gtrsim 10^{10}M_\odot$) every $\log\,M_*$ bin in the 
SC-SAM has an intermediate component. {\color{red} While the $\log\,\mathrm{SSFR}$ 
distributions in the bottom right panels of Figures~\ref{fig:pssfr_gmm_inst} 
and~\ref{fig:pssfr_gmm_100myr} and the BIC values illustrate the benefit of the 
GMM with an intermediate SF component, these are accentuated by the broader 
distribution of the low SF population. Despite these differences between the 
hydrodynamic simulations and the SC-SAM, all of the simulations have a low SF 
component throughout their stellar mass 
range, even at $M_* < 10^9M_\odot$. We discuss these low $M_*$ low SF galaxies 
in further detail later in this section.}

Another set of parameters we infer from our GMM fitting is the weight of the 
GMM components: $\pi_i$ in Eq.~\ref{eq:gmm}. These weights correspond to 
the fractional contribution of the different populations. For example, the 
weight of the low SF component loosely corresponds to the quiescent 
fraction~\citep[\emph{e.g.}][]{borch2006, bundy2006, iovino2010, geha2012, hahn2015}. 
In Figure~\ref{fig:kandinsky} we present the fractional contribution, as a 
function of stellar mass, for all the components from our GMM fitting : SFS 
(blue), low SF (orange), $\mathrm{SFR}{=}0$ galaxies (red), intermediate SF 
(green), and high SF (purple). We present the uncertainties of $\pi_i$s, 
estimated through bootstrap resampling, in Figure~\ref{fig:fcomp_uncertainty}. 
For every simulation, a significant fraction of galaxies have SFR$=0$. 
In hydrodynamic simulations, a galaxy with $\mathrm{SFR}{=}0$ can have an 
SFR below the resolution limit, or have a \emph{true} $\mathrm{SFR}{=}0$ on 
the measured timescales (Appendix~\ref{app:zerosfr}). In the SC-SAM, all 
the $\mathrm{SFR}{=}0$ galaxies have truly no star formation on the 
timescales that we present here. Therefore, in both hydrodynamic 
and SAM simulations, the SFR$=0$ galaxies can be considered quiescent. 
Moreover, we confirm that SFR resolution does not significant impact the 
fraction contributions of Figures~\ref{fig:kandinsky} 
and~\ref{fig:fcomp_uncertainty} (see Appendix~\ref{app:zerosfr} and 
Figure~\ref{fig:mlim_fcomp}).

The fractional contributions of the GMM components in Figures~\ref{fig:kandinsky} 
and~\ref{fig:fcomp_uncertainty} reveal significant disagreements between our 
simulated galaxies and trends established from observations---especially the 
hydrodynamic simulations. For instance, if we treat the low SF component and the 
$\mathrm{SFR}{=}0$ galaxies as quiescent (orange and red in Figure~\ref{fig:kandinsky}),
\emph{we find little stellar mass dependence in the quiescent fraction of our
hydrodynamic simulations, unlike the quiescent fraction measurements 
of isolated SDSS galaxies}~\citep{baldry2006,peng2010,hahn2015}. 
Meanwhile, at $M_* > 10^9M_\odot$ the SC-SAM is consistent with the 
SDSS isolated galaxy quiescent fractions and in agreement with previous 
SC-SAM quiescent fraction comparisons~\citep{brennan2015,brennan2017,pandya2017}.

Furthermore, for some of the hydrodynamic simulations in Figure~\ref{fig:kandinsky}
(Illustris $100\,\mathrm{Myr}$ SFRs, EAGLE, and {\sc Mufasa} instantaneous SFRs)
we find high quiescent fractions (${\sim}0.4$) in strong contrast with 
observations~\citep{baldry2006,peng2010,hahn2015}. In fact, \emph{all the 
simulations, even the SC-SAM, have non-negligible (${\gtrsim}10\%$) quiescent 
fraction at $M_* < 10^9 M_\odot$ contrary to the stellar mass lower bound 
of ${\sim}10^9M_\odot$ for isolated/central quiescent galaxies set by 
observations}~\citep{geha2012}. 
%In fact, the quiescent fraction of EAGLE, for instance, is ${\sim}0.4$ at $ 10^9M_\odot$. Illustris with $100\,\mathrm{Myr}$ SFRs and {\sc Mufasa} with  instantaneous SFRs also have similarly high quiescent fractions. Every  simulation panel of Figure~\ref{fig:kandinsky}, even the SC-SAM, has a  non-negligible (${\sim}10\%$) quiescent fraction at $10^9M_\odot$. 
%, which all find significant stellar mass dependence in the quiescent/red fraction of  isolated galaxies in SDSS. In this regards, the quiescent fraction of the 

One possible explanation for the significant fraction of low SFR galaxies
at low $M_*$ in the hydrodynamic simulations is misclassification of 
``splashback'' (or ``blacksplash'' or ``ejected'') galaxies as centrals. 
Splashback galaxies are satellite galaxies that have orbited outside 
the virial radii of its host halo after having passed through 
it~\citep[\emph{e.g.}][]{mamon2004,gill2005,wang2009a,wetzel2014}.
{\color{red} The SC-SAM is {\em not} subject to this misclassification 
because subhalos are not tracked after mergers, so by construction the 
model does not have splashback galaxies. To test whether splashbacks 
impact our results for our hydrodynamic simulations, we adjust our central
galaxy selection criteria in Section~\ref{sec:central} to exclude 
any centrals with a more massive halo within three virial radii of it. 
When we use this stricter central classification and measure the
SFS and other GMM components, we find \emph{no} significant change to 
the SFS fits or the fractional contributions of the GMM components. 
We also find no significant changes to our results when we restricting 
ourselves to galaxies with no ``luminous'' neighbors within 
$1.5\,\mathrm{Mpc}/h$ --- similar to the~\cite{geha2012} criteria.
Therefore, we conclude that the significant fraction of low SFR and 
low $M_*$ galaxies is not caused by misclassification of centrals.

%Our data-driven GMM SFS fitting provides a principled way to compare 
%galaxy populations through the major features in the data-space. Through
%such comparisons, we've so far revealed a number significant discrepancies
%among the central galaxies of our simulations. However, these comparisons
%are among ``theoretical'' predictions of galaxy properties from simulations. 
%At the end of the day, both hydrodynamic simulations and SAMs aim to 
%reproduce the observations, which requires comparing simulations to observations.

Taking a step back, we emphasize that this discrepancy between the 
simulations and observations must be taken with a grain of salt and
our comparison is not an apples-to-apples comparison. For instance,
in \cite{geha2012} low SF/quiescent galaxies are classified based on 
a $H\alpha$ emission and $D_n 4000$ criteria --- different than in 
our simulations. Even the central (isolation in~\citealt{geha2012}) 
criteria, in detail, is different than the analogous criteria 
above. More broadly, the comparisons we present in this paper are 
among simulations and therefore are based on ``theoretical'' predictions 
of galaxy properties from simulations. Many factors make it difficult 
to robustly extend this comparison to observations. 

For example, SFRs and $M_*$, the 
main galaxy properties considered in this paper, in simulations can be
directly measured either using star or gas particles in the simulations.  
In observations, even the SFR alone is estimated from SFR indicators 
such as $H\alpha$ flux, $D_n 4000$, or UV brightness measurements. 
While they serve as estimates of the SFRs, as \cite{speagle2014} find even 
for the same SDSS galaxies, different SFR indicators can produce large 
discrepancies in the slope and amplitude of the SFS. Furthermore, a 
consistent comparison to observations requires a thorough understanding of 
the selection effects that come with the observed galaxy sample. These 
effects are difficult to propagate into SFR and $M_*$ space of simulations. 

%classifications in \cite{geha2012} are defined differently than in the simulations. 
%\cite{geha2012} classifies as a galaxy as quenched if it
%does not have an $H\alpha$ emission and based on a $D_n 4000$ criteria. 
%Furthermore, the isolation criteria of \cite{geha2012} is more 
%stringent than the group finder at these mass scales. 

Therefore, while we can see from Figures~\ref{fig:sfmsfit_inst}, 
\ref{fig:sfmsfit_100myr}, and \ref{fig:kandinsky} that there are significant
discrepancies between the simulations and observations in the slope and 
amplitude of the SFS and also the fractional contribution of the best-fit 
GMM components, we reserve further scrutiny to the next paper in our 
series: (Starkenburg et al. in prep.). In this next paper, instead of
comparing the ``theoretical'' galaxy properties, we forward model galaxy 
spectra and photometry using the star formation histories from the
simulations, make observationally motivated measurements that reflect 
SFR and $M_*$ on the synthetic spectra and photometry, and conduct a 
quantitative, apples-to-apples, comparison of the simulations to observations.}

In this section, we demonstrate that our SFS fitting provides additional 
features besides the SFSs, to compare different galaxy samples. Extra 
components identified by our fitting method offer insights into the 
distinct galaxy subpopulations of our simulations. Based on the non-SFS 
components/populations, we find consistency among the hydrodynamic
simulations but significant discrepancies with the 
SC-SAM. Moreover, we find that all of the simulations have a significant
fraction of low SFR central galaxies at $M_*\,{\lesssim}\,10^9M_\odot$.
Furthermore, the hydrodynamic simulations, at even $M_*\,{\lesssim}\,10^{11}M_\odot$, 
do not reproduce the quiescent fractions 
from the literature or their stellar mass dependence. 

%The ultimate goal of 
%these simulations, however, is to reproduce the observations, which means
%we ultimately want to compare them to observations. In this section, we 
%extend our GMM SFS fitting to the observed galaxy sample from SDSS and NSA
%and compare our simulated galaxies to observation.

%We first focus on the best-fit SFS of our SDSS and NSA central galaxies
%in the top right panels of Figure~\ref{fig:sfmsfit_inst} and~\ref{fig:sfmsfit_100myr}. 
%Similar to the Illustris, EAGLE, and SC-SAM simulations, the SFR of 
%the SFS monotonically increases with $M_*$. The SFS has no turnover like
%{\sc Mufasa}'s. Overall, the SDSS and NSA SFS is shallower and has a lower amplitude 
%than the SFSs of our simulations. 
%\begin{itemize}
%  \item \todo{Why are there significant discrepancies in the SFS slopes 
%  of the simulation versus observations? What are the main subgrid processes
%  that dictate the slope of the SFS?} Save this discussion for paper 2.  
%  \item \todo{Is there selection bias in the low mass end of the observations?} 
%  More reason why we should reserve comparison!
%\end{itemize}
%
%Looking beyond the SFS, the other components of the SDSS and NSA central 
%galaxy population further highlight the disagreements between the 
%simulations and observation from the previous section (Figure~\ref{fig:kandinsky}). 
%At $M_* \lesssim 10^9M_\odot$, we find no quenched population, in agreement 
%with~\cite{geha2012} and in disagreement with the simulations. Furthermore, 
%in agreement with measurements in the literature, we find significant $M_*$
%dependence in the quiescent fraction, which the hydrodynamic simulations do 
%not find. Aside from the low stellar mass quenched population, the components 
%of the SC-SAM simulation are in good agreement with the SDSS and NSA 
%components. They both identify star-burst populations at low masses and 
%significant transitioning populations from $M_* > 10^{10}M_\odot$.  

%\subsection{Comparing to Observations}
%The comparison we make between our simulations and observations so far in 
%this section only scratches the surface. However, we deliberately refrain 
%from a more detailed comparison due to the limitations of such a comparison. 
%The main bottleneck stems from the fact that the SFRs and $M_*$, the main
%galaxy properties considered in this paper, are defined and measured 
%differently in simulations versus observations. 
%Therefore, for further scrutiny we require a more apples to apples 
%comparison. In the next paper we will bring this comparison on a more equal 
%footing by deriving star formation rates from mock observations of the 
%simulated galaxies (Starkenburg et al. in prep.).

%The SDSS samples show a SFR-$M_*$ relation that is best fitted by \emph{one} Gaussian at lower masses, unlike any of the simulations. This is noteworthy even though the low-mass sample is far from volume-complete and especially many low-SFR galaxies may be missed at lower masses.
%\item[-] The observations show a very well defined, and for higher mass bins quite narrow, low-SFR population, with a small intermediate third population for only three mass bins. The low-SFR population is more narrow and generally with the mean at slightly lower SFR than for all three hydro simulations. It is located in between the intermediate and low-SFR populations of the SAM.

%\todo{TKS: Compare to Speagle2014 combined dataset fits and discuss with respect to the large discrepancy they note in MS-slopes for the SDSS sample of $~0.4$ dex. This is partly due to using different SFR indicators (nice segue to the 2nd and 3rd papers), but not completely}
%comparison to observations are complicated by a number of factors.  First the SFRs and stellar masses of observations are defined by kcorrect  and emission lines and dn4000. While our SFR timescales were chosen to best  reflect the observed SFR, it's still not an apples to apples comparison.  Second, the stellar mass and SFRs of observed galaxies have uncertainties  associated to them. If unaccounted for, these uncertainties will impact the  components recovered from the GMM fitting and ultimately the SFS fits. 
%All of the points in this subsection are severely affected by the different ways that SFR (and $M_*$) are measured in the observations and the simulations. 

\section{Summary and Conclusions} \label{sec:summary}
The Star Forming Sequence is a well-defined feature in the galaxy 
property space that provides a key feature to consistently compare
galaxy populations in simulations and observations.
%of both simulations and observations. It provides  a key feature to consistently compare the galaxy populations of  simulations and observations. 
%provides a key feature in galaxy property  space to consistently compare the galaxy populations in simulations and observations. 
Such comparisons are crucial for validating our theories of
galaxy formation and evolution. However, they face two main challenges: 
the lack of a consistent data-driven method for identifying the SFS and 
the discrepancies in methodology for deriving galaxy properties such as 
SFR and $M_*$. In this paper, we address the former by presenting 
a flexible data-driven method for fitting the SFS. 

Our method takes advantage of Gaussian Mixture Models to fit the SFR
distributions in stellar mass bins and Bayesian Information Criteria 
for model selection. This data-driven approach allows us to robustly 
fit the SFR-$M_*$ relation of galaxy populations and identify the SFS, 
while relaxing many of the assumptions and hard cuts that go into 
previous methods. Furthermore, it allows us to flexibly identify the 
SFS over  a wide range of star formation and to stellar masses down to 
$M_*{\sim}10^{8}M_\odot$. Finally, our method also allows us to identify 
subpopulations of galaxies, beyond the SFS, that correspond to the 
quiescent, transitioning, and star-burst galaxy populations. 

Next, we apply our SFS fitting method to the central galaxies of the 
Illustris, EAGLE, and {\sc Mufasa} hydrodynamic simulations and the Santa 
Cruz Semi-Analytic Model. The central galaxies are identified in each 
simulation using% an extended version of 
the \cite{tinker2011} group 
finder and have \emph{consistently} derived $M_*$ and SFRs on instantaneous 
and $100\,\mathrm{Myr}$ timescales. For reference, we also apply our 
method to observations from the SDSS Data Release 7 and the NASA Sloan 
Atlas. Comparing the resulting SFSs and other components from our simulations 
and observations, we find the following results:

\begin{itemize}
\item The SFRs among the best-fit SFSs of Illustris, EAGLE, {\sc Mufasa}, 
and SC-SAM have order of magnitude discrepancies throughout the stellar mass 
range $10^{8.5} M_\odot < M_* < 10^{11} M_\odot$ with little mass
dependence in the discrepancies. Meanwhile the width of the best-fit SFSs 
are consistent among the simulations and in agreement with observations.

\item The best-fit GMMs reveal that the hydrodynamic simulations are 
mainly dominated by the SFS and low SF (quiescent) components. Meanwhile,
the SC-SAM is composed of a substantial fraction of galaxies between the 
SFS and low SF components at high masses ($M_* > 10^{10}M_\odot$) and above 
the SFS at low masses ($M_* < 10^{10}M_\odot$).

\item The quiescent fractions of the hydrodynamic simulations, estimated 
from the low SF components of the best-fit GMMs and $\mathrm{SFR}{=}0$ 
galaxies, have little stellar mass dependence and are inconsistent with 
the SC-SAM and observations. Moreover, in all of our simulations, we find 
a significant fraction of low SF (quiescent) galaxies at $M_*{<}\,10^9M_\odot$, 
in conflict with the $M_*$ lower bound of isolated quiescent galaxies in 
\cite{geha2012}.
\end{itemize}

%Overall, the comparison we conduct using the SFS fitting method  we present in this paper, reveals significant discrepancies among the  central galaxies of Illustris, EAGLE, {\sc Mufasa}, and SC-SAM.
%Using the SFS fitting method we present in this paper, we find significant discrepancies among the central galaxy populations of Illustris, EAGLE,  {\sc Mufasa}, and SC-SAM.  
By using the SFS fitting method we present in, this paper along with a 
consistent treatment of our simulated galaxies, we demonstrate that the 
central galaxies of Illustris, EAGLE,  {\sc Mufasa}, and SC-SAM have significant
discrepancies in their properties. Although we refrain from a detailed 
comparison with observations, we also find significant discrepancies 
between the simulations and established trends in observations. These
discrepancies, which previous comparisons had failed to identify, underscore 
the importance of a consistent data-driven approach for accurately comparing 
galaxy populations.
%consistent and principled comparisons among galaxy populations.
%how a  consistent comparison of galaxy populations can reveal detailed discrepancies. 

Furthermore the results of this paper illustrate how differences in the 
sub-grid physics of our simulations propagate into large discrepancies 
in the properties of their galaxy populations. In other words, a 
consistent treatment of simulations and observations combined with a 
flexible data-driven comparison, like the one we present in this work, can 
be used test the sub-grid physics models of our simulations and ultimately 
constrain the physics of galaxy formation and evolution.
%This means that the SFS  fitting method, combined with a consistent treatment of simulations and  observations, can be used to test our  sub-grid physics models and 
%ultimately constrain the physics of galaxy formation and evolution.
This is the exact approach we will present in the subsequent paper of our 
series---Starkenburg et al. in prep. %We construct mock observational spectra  of simulated galaxies and measure galaxy properties in the same way for both  simulations and observations. Then using the SFS fitting we present in this  paper, we compare simulations to observations in detail and test the physics  that go in our simulations. 
%provides an effective method for  constraining the physics of galaxy formation and evolution.

\section*{Acknowledgements}
It's a pleasure to thank
	Shirley~Ho, 
    John~Moustakas,
	Emmanuel~Schaan, 
    \todo{more acknowledgements here} 
for valuable discussions. 
This material is based upon work supported by the U.S. Department
of Energy, Office of Science, Office of High Energy Physics, under
contract No. DE-AC02-05CH11231. The Isolated and Quiescent galaxies (IQ) 
collaboratory thanks the Flatiron Institute for hosting the collaboratory 
and its meetings. The Flatiron Institute is supported by the Simons Foundation.
\appendix
\counterwithin{figure}{section}

\section{Earlier Comparisons of the Star Forming Sequence} \label{app:literature}
Earlier SFS comparisons in the literature overall report agreement among 
simulations and observations~\citep[\emph{e.g.}][]{genel2014, somerville2015, sparre2015, schaye2015, bluck2016, dave2016}. 
This agreement is particularly evident in the seminal comparison in 
\cite{somerville2015b} (Figure 5). As noted in \cite{somerville2015b}, 
the SFSs compiled in the comparison are derived inconsistently with some 
applying a star-forming galaxy selection cut (\emph{e.g.} SSFR cut) and 
other not applying any cut. {\em Inconsistency in measuring the SFS can 
produce misleading agreement among simulations.} 

%%%%%%%%%%%%%%%%%%%%%%%%%%%%%%%%%%%%%%%%%%
% Figure 
%%%%%%%%%%%%%%%%%%%%%%%%%%%%%%%%%%%%%%%%%%
\begin{figure*}
\begin{center}
\includegraphics[width=0.65\textwidth]{figs/Catalog_SFS_lit.pdf} 
\caption{The SFSs of Illustris, EAGLE, {\sc Mufasa}, and SC-SAM central 
galaxies, where we measure the SFSs using multiple methods as in Figure 5 
of \cite{somerville2015b} (left panel) and using the same method (right panel). 
In the left panel, we measure the SFSs by taking the median SFR in 
a $M_*$ bin with no selection cuts for Illustris and {\sc Mufasa}
and by taking the median SFR after a SSFR > $10^{-11}\,yr^{-1}$ cut 
for EAGLE and SC-SAM. In the right panel, we measure the SFSs by taking 
the median SFR after a SSFR > $10^{-11}\,yr^{-1}$ cut for all four simulations.
The difference between the two panels illustrate that \emph{the agreement 
found in the left panel, and similarly in \cite{somerville2015b},
is mainly driven by the difference in methods used to measure SFSs.}
} 
\label{fig:likeSD14}
\end{center}
\end{figure*}
%%%%%%%%%%%%%%%%%%%%%%%%%%%%%%%%%%%%%%%%%%

In the left panel of Figure~\ref{fig:likeSD14} we reproduce the SFS comparison of 
\cite{somerville2015b} Figure 5 for our simulations using different methods for measuring the SFS. 
For Illustris and EAGLE, we apply the same methods as the SFSs in \cite{somerville2015b}: 
the median SFR in a $M_*$ bin with no selection cut for Illustris (green) and 
with a SSFR > $10^{-11}\,yr^{-1}$ cut for EAGLE~\citep[red][]{schaye2015}. 
{\sc Mufasa} and the current version of SC-SAM did not exist and were not 
included in \cite{somerville2015b}. Since we're illustrating how inconsistent 
SFS measurements can result in misleading agreement, for {\sc Mufasa} and 
SC-SAM we measure the median SFR with no selection cut and with a SSFR > 
$10^{-11}\,yr^{-1}$ cut respectively. As in Figure 5 of \cite{somerville2015b}, 
we find good agreement among the SFSs of our simulations. 

Instead of measuring the SFSs differently, if we measure the SFS by taking the
median SFR after a SSFR > $10^{-11}\,yr^{-1}$ cut consistently for all our 
simulations, we find discrepancies in the SFSs on the order ${\sim}0.5\,\mathrm{dex}$
(right panel of \ref{fig:likeSD14}). The agreement found in \cite{somerville2015b} 
is mainly driven by the difference in methods used to measure SFSs. 
Furthermore, Figure~\ref{fig:likeSD14} highlights the impact of hard selection cuts 
and the importance of using a consistent methodology for measuring the SFS. 

%In this paper we perform a Gaussian Mixture Model fitting of the SFR-$M_{\star}$ plane and find almost an order of magnitude discrepancy in normalization of the Star Formation Sequence when described by the peak of our most dominant Gaussian component. This is seemingly in contrast to results from the literature that report agreement between simulations and observations ~\citep[\emph{e.g.}][]{ 
%somerville2015, genel2014, sparre2015, schaye2015, bluck2016, dave2016}. This apparent "disagreement" between our and previous results is mostly due to whether the authors applied a cut to select star-forming galaxies before fitting the SFS or not. Arguments can be found for both sides: fitting the SFS using only galaxies above a minimum sSFR or fitting the SFS using all galaxies. Additional disagreement exists in whether to use means, or medians, or when applying a cut-off what that should be. In Fig. \ref{fig:likeSD14} we show an identical figure to the comparison figure (Fig. 5) in \citet{somerville2015} where we apply almost exactly the same selection as the modelers did for that figure for Illustris and EAGLE (we chose a cut of sSFR > $-11$ yr$^{-1}$ instead of the sSFR > $-10.5$ yr$^{-1}$ used in \citet{schaye2015}). Both MUFASA and the current version of the SCSAM were not in existence at the time and for those we chose to either select star-forming galaxies (using a cut of sSFR > $-11$ yr$^{-1}$) or all galaxies, whichever fit best to the Illustris and EAGLE results. The caption of Fig. 5 in \citet{somerville2015} noted that "some of the modelers have applied a cut to select star forming galaxies, but some have not". As is shown in that figure and Fig. \ref{fig:likeSD14}, the simulations agree very well on the slope and normalization of the SFS. 
%However, when we choose to either select the star forming galaxies or all galaxies for all simulations the results are quite different. This can be seen in Fig. \ref{fig:medianSFall}. In both cases selecting the same sample of galaxies results in a disagreement in the normalization of the SFS of > 0.5 dex. This is the main driver of the disagreement that we found in this paper. Additional differences can arise in our method because a GMM models both the peak and the width of the SFS and is more sensitive to the position of the green valley with respect to the SFS. Additionally applying GMM results in intermediate or transitioning and starburst components when those populations are statistically significant. While we also use only central galaxies and apply a group finder to select those centrals, these effects are very minor. 

\section{SFS fitting using Gaussian Mixture Models} \label{app:gmm_pssfr}
{\color{red}
In order to derive the best-fit GMM used in identifying the SFS in 
each $M_*$ bin, we compare GMM fits with $k\leq 3$ components using
their BICs (Section~\ref{sec:sfmsfit}). In Figures~\ref{fig:pssfr_gmm_inst} 
and~\ref{fig:pssfr_gmm_100myr}, we illustrate this comparison among the  
GMMs with $k=1, 2,$ and $3$ (blue, orange, and green) components fit
to the instantaneous SSFR distributions, $P(\log\,\mathrm{SSFR})$, of the 
Illustris, EAGLE, {\sc Mufasa}, and SC-SAM (top to bottom panels) centrals 
in three stellar mass bins: 
$9.2 <\log\,M_*<9.4$, $9.8 <\log\,M_*<10$, and $10.6 <\log\,M_*<10.8$ (left to right). 
The SSFR distributions in Figures~\ref{fig:pssfr_gmm_inst} 
and~\ref{fig:pssfr_gmm_100myr} are derived using instantaneous and 
$100\,\mathrm{Myr}$ SFRs respectively. Galaxies with $\mathrm{SFR}=0$
are represented at the edge of the SSFR distributions with $\mathrm{SSFR}=-13.5$.
In each panel, we also present the BICs and plot every component of the 
GMM fits (dashed) in their respective colors. These figures illustrate
the advantages of the data-driven GMM based fitting and BIC based
model selection used in our SFS fitting.

%%%%%%%%%%%%%%%%%%%%%%%%%%%%%%%%%%%%%%%%%%
% Figure 
%%%%%%%%%%%%%%%%%%%%%%%%%%%%%%%%%%%%%%%%%%
\begin{figure*}
\begin{center}
\includegraphics[width=0.8\textwidth]{figs/Pssfr_GMMcomps_inst.pdf} 
\caption{GMMs with $k=1, 2,$ and $3$ (blue, orange, and green) components fit
to the instantaneous SSFR distributions, $P(\log\,\mathrm{SSFR})$, of the 
Illustris, EAGLE, {\sc Mufasa}, and SC-SAM (top to bottom panels) centrals 
in three stellar mass bins: $[9.2, 9.4]$, $[9.8, 10.]$, and $[10.6, 10.8]$ 
(left to right). For every GMM fit, we plot each component in dash lines 
and list their BICs in the same color. In our SFS fitting, we select the 
GMM with the lowest BIC as the best-fit. This provides a \emph{data-driven 
way of accurately fitting the SSFR distribution while also avoiding 
overfitting}.
} 
\label{fig:pssfr_gmm_inst}
\end{center}
\end{figure*}
%%%%%%%%%%%%%%%%%%%%%%%%%%%%%%%%%%%%%%%%%%
%%%%%%%%%%%%%%%%%%%%%%%%%%%%%%%%%%%%%%%%%%
% Figure 
%%%%%%%%%%%%%%%%%%%%%%%%%%%%%%%%%%%%%%%%%%
\begin{figure*}
\begin{center}
\includegraphics[width=0.8\textwidth]{figs/Pssfr_GMMcomps_100myr.pdf} 
\caption{Same as Figure~\ref{fig:pssfr_gmm_inst} but for the $100\,\mathrm{Myr}$
SSFR distributions.} 
\label{fig:pssfr_gmm_100myr}
\end{center}
\end{figure*}
%%%%%%%%%%%%%%%%%%%%%%%%%%%%%%%%%%%%%%%%%%


For instance, in most of the highest $M_*$ bin (right panels in both figures) 
the $k=1$ GMM fits do not reflect the clearly bimodal SSFR distributions.
In these cases, the $\mathrm{BIC}_{k=1}$ is significantly larger than
the BICs of $k = 2$ and $3$ component GMM fits so our BIC based model 
selection favors GMMs with more than one component. In fact, GMMs with
more components are more flexible and generally can better fit the underlying 
distribution. However as the EAGLE and {\sc Mufasa} $9.8 <\log\,M_*<10$ 
bins of the figures illustrate, our BIC based model selection does not 
always favor the higher $k$ GMM fits. Although the $k=3$ GMMs have the 
lowest $\chi^2$ in these panels, because of the penalty term for the 
number of model parameters, our BIC criteria favors the $k=2$ GMMs.
According to the BICs, the $k=3$ GMMs in these panels overfit the 
SSFR distributions.

In a number of stellar mass bins, the best-fit GMMs have more than one 
$\log\,\mathrm{SSFR} > -11$ components. In these cases the 
$\log\,\mathrm{SSFR} > -11$ component with the highest weight is identified
as the SFS component (Section~\ref{sec:sfmsfit}). We consider other 
components, depending on their mean, as intermediate or high SF components in 
Section~\ref{sec:beyondsfms}. The SC-SAM in particular has high SF
components at $M_* \lesssim 10^{10}M_\odot$ (bottom left and center panels
of Figures~\ref{fig:pssfr_gmm_inst}  and~\ref{fig:pssfr_gmm_100myr}).
In these cases, the SSFR distribution is not well described by a single 
log-normal distribution. Instead the distribution is asymmetric with a 
heavier tail on the more star forming end of the distribution. An
extra component to account for the heavier tail improves the fit more 
than the penalty term, giving us the high SF components.}

%So far we have assumed that one, two, or three populations would describe galaxies in the SFR--$M_*$ plane well. We have checked whether the possibility of adding more than three Gaussians in the GMM shows that more populations are present. This is the case only for the Santa Cruz semi-analytic model, where in a number of bins the presence of 4, 5, or (in one case) 6 Gaussians is preferred. In almost all cases the added Gaussians have low weight and form effectively additional intermediate populations. For both Illustris and EAGLE there is only one mass bin where a fourth population would be preferred. For all other mass bins in all the hydrodynamic simulations, and all mass bins in the SDSS 3 or less Gaussians are preferred.

%In addition to identifying the SFS, the GMM fitting method described above can  be extended to describe the entire SFR-$M_*$ relation using a two-dimensional  GMM. In Figure~\ref{fig:2dgmm}, we compare the instantaneous SFR to $M_*$ relation  of central galaxies in the EAGLE simulation with the best-fit two-dimensional GMM.  The over-plotted shaded ellipses represent the two-dimensional Gaussian components of the best-fit GMM. Overall the best-fit 2D GMM captures the features in the  EAGLE SFR-$M_*$ relation. It also provides a straightforward way of comparing  different SFR-$M_*$ relations. However, as Figure~\ref{fig:2dgmm} illustrates,  specifically identifying the SFS using the two-dimensional model is more  challenging. Therefore in this paper, we do not discuss the 2D GMM further. 

\section{SFR Resolution Effects in Hydrodynamic Simulations} \label{app:zerosfr}
In our analysis, we consistently derive SFRs for all of our simulated
galaxies on two timescales: instantaneous and averaged over 
$100\,\mathrm{Myr}$ (Section~\ref{sec:galsims}). For our hydrodynamic 
simulations, SFR averaged over $100\,\mathrm{Myr}$ is derived using 
the formation times of the star particles in the simulation, which 
means that both the mass and temporal resolutions of the simulations 
impact the $100\,\mathrm{Myr}$ SFR. In Illustris, EAGLE, and {\sc Mufasa},
their $100\,\mathrm{Myr}$ SFRs have resolutions of 
$\Delta_\mathrm{SFR} = 0.016$, $0.018$, and $0.182\,M_*/\mathrm{yr}$, 
corresponding to baryon particle masses of $1.6 \times 10^6\ M_{\sun}$, 
$1.8 \times 10^6\ M_{\sun}$, and $1.82 \times 10^7\ M_{\sun}$, respectively. 

%%%%%%%%%%%%%%%%%%%%%%%%%%%%%%%%%%%%%%%%%%
% Figure 
%%%%%%%%%%%%%%%%%%%%%%%%%%%%%%%%%%%%%%%%%%
\begin{figure*}
\begin{center}
\includegraphics[width=0.8\textwidth]{figs/Pssfr_res_impact.pdf} 
\caption{The impact of SFR resolution on the SSFR distribution, 
$P(\log\,\mathrm{SSFR})$, in two stellar mass bins of our hydrodynamic 
simulations:
Illustris (left), EAGLE (center), and MUFSAS (right). We plot the 
$P(\mathrm{SSFR})$ distributions using the $100\,\mathrm{Myr}$
SFRs \emph{with} resolution effects in black. In blue, we plot the 
$P(\log\,\mathrm{SSFR})$ distributions where the SFRs of the galaxies
are sampled uniformly within the SFR resolution range 
($[\mathrm{SFR}_i, \mathrm{SFR}_i+\Delta_\mathrm{SFR}]$). The 
uncertainties for the blue $P(\mathrm{SSFR})$s are estimated from 
re-sampling the SFR of each galaxy based on the SFR resolution. 
At low stellar masses (top) the SFR resolution significantly impacts 
the star-forming end of $P(\mathrm{SSFR})$s. At higher stellar masses, 
although the SFR resolution impacts the $P(\mathrm{SSFR})$s, the effect 
is limited to $\log\,\mathrm{SSFR} < -11$.
} 
\label{fig:sfrres_pssfr}
\end{center}
\end{figure*}
%%%%%%%%%%%%%%%%%%%%%%%%%%%%%%%%%%%%%%%%%%

% describe the impact of resolution effect more concretely 
For galaxies with high $100\,\mathrm{Myr}$ SFR, the resolution 
$\Delta_\mathrm{SFR}$ is relatively small compared to their SFRs and
therefore it does not have a significant impact. However for low SFR 
galaxies, the resolution effect is more significant. At the lowest SFR 
end, galaxies that, without the resolution effect, would have SFR ranging 
$0 < \mathrm{SFR} < \Delta_\mathrm{SFR}$, have $\mathrm{SFR}{=}0$ with 
the resolution effect. These galaxies are thereby not included in the 
$\log\,\mathrm{SFR}$--$\log\,M_*$ plane or the SFS fitting. In
Figure~\ref{fig:sfrres_pssfr}, we present the impact of excluding these
galaxies and the overall resolution effect on the $P(\log\,\mathrm{SSFR})$ 
distributions of our hydrodynamic simulations in two stellar mass bins. In 
black, we plot the $P(\log\,\mathrm{SSFR})$ distributions using the 
$100\,\mathrm{Myr}$ SFRs \emph{with} resolution effects
(\emph{excluding} galaxies with $\mathrm{SFR}{=}0$). In blue, we plot the 
$P(\log\,\mathrm{SSFR})$ distributions of \emph{all} galaxies where
$\mathrm{SFR}'_i$ of each galaxy sampled uniformly within the SFR resolution range, 
$[\mathrm{SFR}_i, \mathrm{SFR}_i+\Delta_\mathrm{SFR}]$. Uncertainties 
for the blue $P(\log\,\mathrm{SSFR})$s are derived from repeating this 
SFR sampling $100$ times. For the low $M_*$ bins (top), the SFR resolution 
affects the $P(\log\,\mathrm{SSFR})$s well above $\log\,\mathrm{SSFR}{=}-11.$ 
on the star-forming end of the distribution. Meanwhile, the impact at higher
$M_*$ (bottom), is limited to the low SSFR end. 

%%%%%%%%%%%%%%%%%%%%%%%%%%%%%%%%%%%%%%%%%%
% Figure 
%%%%%%%%%%%%%%%%%%%%%%%%%%%%%%%%%%%%%%%%%%
\begin{figure*}
\begin{center}
\includegraphics[width=0.75\textwidth]{figs/Mlim_res_impact.pdf} 
\caption{The resolution effect of $100\,\mathrm{Myr}$ SFRs in the 
hydrodynamic simulations (Illustris, EAGLE, and {\sc Mufasa}) impact 
SFS fitting at low stellar mass. In black we plot the best-fit SFS 
with the resolution effects. In orange we plot the best-fit SFS where 
the SFR for each galaxy is sampled uniformly within the resolution range: 
$\mathrm{SFR}_i^{\prime} \in [\mathrm{SFR}_i, \mathrm{SFR}_i +
\Delta_\mathrm{SFR}]$). SFR resolution impacts SFS fits at low 
stellar mass. Based on the discrepancy between the fits, we determine 
stellar mass limits above which the SFR resolution does {\em not} 
significantly impact ($< 0.2\,\mathrm{dex}$) the SFS fit. For Illustris, 
EAGLE, and {\sc Mufasa} this corresponds to $\log M_\mathrm{lim} = 8.4, 8.4$, 
and $9.2$, respectively.} 
\label{fig:mlim_res}
\end{center}
\end{figure*}
%%%%%%%%%%%%%%%%%%%%%%%%%%%%%%%%%%%%%%%%%%
In order to better quantify the impact of the SFR resolution effect
on our SFS fitting, in Figure~\ref{fig:mlim_res} we compare the SFS fits 
using $100\,\mathrm{Myr}$ SFRs \emph{with} resolution effects (black) to 
the SFS fits using $100\,\mathrm{Myr}$ SFRs sampled uniformly within the 
SFR resolution range (orange; 
$\mathrm{SFR}_i^{\prime} \in [\mathrm{SFR}_i, \mathrm{SFR}_i + \Delta_\mathrm{SFR}]$). 
The uncertainties of our SFS fits in black are calculated using bootstrap 
resampling (Section~\ref{sec:sfmsfit}). In agreement with 
Figure~\ref{fig:sfrres_pssfr}, we find that the SFR resolution 
significantly impacts SFS fitting at low $M_*$. Moreover, using the 
comparison of Figure~\ref{fig:mlim_res}, we determine the stellar mass 
limit above which the SFR resolution does {\em not} significantly impact 
the SFS fitting --- \emph{i.e.} the shift in best-fit SFS is below 
$0.2\,\mathrm{dex}$. For Illustris, EAGLE, and {\sc Mufasa} we determine 
$\log\,M_\mathrm{lim} = 8.4, 8.4$, and  $9.2$, respectively. {\color{red} 
For EAGLE, where we have a higher resolution box ($8\times$ higher baryon 
mass resolution) available, we further confirm that the SFS is not significantly impacted above $\log\,M_\mathrm{lim}$.}

%%%%%%%%%%%%%%%%%%%%%%%%%%%%%%%%%%%%%%%%%%
% Figure 
%%%%%%%%%%%%%%%%%%%%%%%%%%%%%%%%%%%%%%%%%%
\begin{figure*}
\begin{center}
\includegraphics[width=0.7\textwidth]{figs/GMMcomp_comp_res_impact.pdf}
\caption{Fractional contributions, $\pi_i$s, of the best-fit GMM components 
from our SFS fitting for our hydrodynamic simulations (Illustris, EAGLE, and 
{\sc Mufasa}) where we uniformly sample the $100\,\mathrm{Myr}$ SFRs within
the SFR resolution range --- $\mathrm{SFR}_i^{\prime} \in [\mathrm{SFR}_i, \mathrm{SFR}_i + 
\Delta_\mathrm{SFR}]$. In the bottom panels, we plot the corresponding bootstrap 
uncertainties. In comparison to Figures~\ref{fig:kandinsky} 
and~\ref{fig:fcomp_uncertainty}, we find SFR resolution has no 
significant impact on the qualitative results in Section~\ref{sec:beyondsfms}.} 
\label{fig:mlim_fcomp}
\end{center}
\end{figure*}
%%%%%%%%%%%%%%%%%%%%%%%%%%%%%%%%%%%%%%%%%%

In addition to its effect on the SFS fits, we also examine the impact of 
SFR resolution on our results regarding the non-SFS components of our GMM 
fitting (Figures~\ref{fig:kandinsky} and~\ref{fig:fcomp_uncertainty}). In 
Figure~\ref{fig:mlim_fcomp} we present the fraction contributions (top; $\pi_i$) 
of the best-fit components and their bootstrap uncertainties (bottom) for 
the Illustris, EAGLE, and {\sc Mufasa} simulations, where we uniformly 
sample the SFRs within the SFR resolution range (same as above). Aside from 
no longer having a $\mathrm{SFR}{=}0$ component, due to the SFR sampling, we 
find no significant change from the $\pi_i$s of Figures~\ref{fig:kandinsky} 
and~\ref{fig:fcomp_uncertainty}) and, thus, the results of Section~\ref{sec:beyondsfms}. 

\bibliographystyle{aasjournal}
\bibliography{paper1}
\end{document}